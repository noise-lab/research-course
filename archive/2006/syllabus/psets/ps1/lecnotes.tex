
%% Please use the following conference proceedings as
%% references:

%% \begin{itemize}
%% \itemsep=-1.5pt
%% \item Networking: {\em Proceedings of ACM SIGCOMM}, September 2006
%% \item Systems: {\em Proceedings of the 7th USENIX Symposoim on Operating
%%   Systems Design and Implementation (OSDI)}, November 2006
%% \item Security: {\em IEEE Symposium on Security and Privacy}, May 2006
%% \item Theory: 
%% \item Artificial Intelligence:
%% \end{itemize}

%% {\bf Select and defend papers.}
%% After selecting your two papers, write a one-page summary of each paper
%% that provides (1) A short summary of the paper; (2) Why you think it
%% satisfies the criteria for good research.


\section{Criteria for ``Good'' Research Problems}

One of the first tasks that a researcher must do---not to mention one of
the most challenging ones---is finding the right problem or problems to
work on.  What makes a great research problem?  

First, it's important to realize that there's not necessarily a
``right'' answer to this question, and that research is often a matter
of taste.  Second, it's often difficult to tell whether a research
problem is truly great at first cut (see today's reading for examples
where reviewers ``got it wrong'' the first time around).

Nevertheless, while there are no hard and fast criteria, great research
problems tend to share many of the following characteristics.  Here are
some of our thoughts on what makes good research; when perusing
conference proceedings, it may help you to keep some of the following
criteria in mind when selecting papers:
\begin{itemize}
\item {\em The problem is difficult.}  Life is too short to solve easy
  problems. 
\item {\em Solving the problem creates new knowledge.} Recognizing the
  difference between research and ``a simple matter of engineering'' is
  important.  Many problems are difficult, but if an army of programmers
  could solve the problem with what we know today, it's probably not
  research.
\item {\em The problem changes conventional thinking.} While there is
  value in confirming conventional wisdom, great research typically
  offers surprising results, creates a new way of thinking about or
  approaching problems.
\item {\em The problem has a solution.} Not every paper needs to solve a
  problem completely (in fact, you'll go looking for future work as part
  of this assignment).  Still, while posing a good question is
  paramount, the solution or finding presented in a paper should
  represent a significant advance beyond the previous body of
  knowledge.  As a Ph.D. student, since you ultimately need to graduate,
  it's good to pick problems that have some solution!
\end{itemize}
\noindent
Needless to say, it's difficult when first starting out on a problem do
determine whether it has these characteristics, but it is often patently
clear when a problem does {\em not} have them.

