\documentclass[11pt]{article}

\usepackage{epsf}
\usepackage{epsfig}
\usepackage{url}
\usepackage{hw}

\newcommand{\newc}{\newcommand}

\newc{\code}[1]{{\tt #1}}
\newc{\func}[1]{{\em #1\/}}

\newc{\be}{\begin{enumerate}}
\newc{\ee}{\end{enumerate}}

\newc{\bi}{\begin{itemize}}
\newc{\ei}{\end{itemize}}

\newc{\bd}{\begin{description}}
\newc{\ed}{\end{description}}

\newc{\ov}[1]{$\overline{#1}$}
\newc{\instr}{\tt}

\newc{\doublespace}{\renewcommand{\baselinestretch}{1.5}}

\newcommand{\figref}[1]{Figure~\ref{#1}}
\newcommand{\tref}[1]{Table~\ref{#1}}

% Captioned table
\newc{\tbl}[3]{
        \begin{table}[htb]
                \centering
                #1
                \caption{#3}
                \label{#2}
        \end{table}
}

% Input a table.
\newcommand{\dblfig}[3]{
        \begin{figure}[htb]
		\centering
                \input{#1}
                \caption{#3}
                \label{#2}
        \end{figure}
}

\newcommand{\ddblfig}[4]{
        \begin{figure}[htb]
		\hspace{-0.1in}
                \psfig{figure=#1,width=0.45\textwidth}
                \caption{#3}
                \label{#2}
        \end{figure}
}

% Figure with no caption
\newcommand{\nofig}[2]{
        \begin{figure}[htb]
                \centering
                \psfig{figure=#1}
                \label{#2}
        \end{figure}
}

% Whole page figure
\newcommand{\schfig}[3]{
        \begin{figure}[p]
                \centering
                \psfig{figure=#1,height=7in}
                \caption{#3}
                \label{#2}
        \end{figure}
}

% Small figure
\newcommand{\sfig}[3]{
        \begin{figure}[ltb]
                \centering
               \hspace*{\fill}\rule{\linewidth}{.5mm}\hspace*{\fill}\vspace{3mm}
                \psfig{figure=#1,width=0.4\textwidth}
                \caption{#3}
                \label{#2}
               \vspace{3mm}\hspace*{\fill}\rule{\linewidth}{.5mm}\hspace*{\fill}
        \end{figure}
}

% Medium figure
\newcommand{\mfig}[3]{
        \begin{figure}[ltb]
		\centering
               \hspace*{\fill}\rule{\linewidth}{.5mm}\hspace*{\fill}\vspace{1mm}
                \psfig{figure=#1,height=2.5in}
                \caption{#3}
                \label{#2}
               \vspace{0mm}\hspace*{\fill}\rule{\linewidth}{.5mm}\hspace*{\fill}
        \end{figure}
}

\newcommand{\widefig}[4]{
        \begin{figure*}[htb]
                \centering
               \hspace*{\fill}\rule{\linewidth}{.5mm}\hspace*{\fill}\vspace{5mm}
                \psfig{figure=#1,width=#3}
                \caption{#4}
                \label{#2}
               \vspace{5mm}\hspace*{\fill}\rule{\linewidth}{.5mm}\hspace*{\fill}
        \end{figure*}
}

\newcommand{\mcfig}[4]{
        \begin{figure}[htbp]
                \centering
               \hspace*{\fill}\rule{\linewidth}{.5mm}\hspace*{\fill}\vspace{5mm}
                \psfig{figure=#1,width=#3}
                \caption{#4}
                \label{#2}
               \vspace{5mm}\hspace*{\fill}\rule{\linewidth}{.5mm}\hspace*{\fill}
        \end{figure}
}

\newcommand{\docfig}[3]{
        \begin{figure}[htbp]
               \hspace*{\fill}\rule{\linewidth}{.5mm}\hspace*{\fill}\vspace{5mm}
                \centering
                \psfig{figure=#1,width=#3}
                \label{#2}
               \vspace{5mm}\hspace*{\fill}\rule{\linewidth}{.5mm}\hspace*{\fill}
        \end{figure}
}

% Medium-large figure
\newcommand{\mlfig}[3]{
        \begin{figure}[htb]
                \centering
               \hspace*{\fill}\rule{\linewidth}{.5mm}\hspace*{\fill}\vspace{5mm}
                \psfig{figure=#1,height=3.25in}
                \caption{#3}
                \label{#2}
               \vspace{5mm}\hspace*{\fill}\rule{\linewidth}{.5mm}\hspace*{\fill}
        \end{figure}
}

% Large figure
\newcommand{\lfig}[3]{
        \begin{figure}[p]
                \centering
               \hspace*{\fill}\rule{\linewidth}{.5mm}\hspace*{\fill}\vspace{5mm}
                \psfig{figure=#1,height=5in}
                \caption{#3}
                \label{#2}
               \vspace{5mm}\hspace*{\fill}\rule{\linewidth}{.5mm}\hspace*{\fill}
        \end{figure}
}

% 'gg' figures are the double column versions of the 'g' figures above.
\newcommand{\sfigg}[3]{
        \begin{figure*}[htb]
                \centering
               \hspace*{\fill}\rule{\linewidth}{.5mm}\hspace*{\fill}\vspace{5mm}
                \psfig{figure=#1,height=1.5in}
                \caption{#3}
                \label{#2}
               \vspace{5mm}\hspace*{\fill}\rule{\linewidth}{.5mm}\hspace*{\fill}
        \end{figure*}
}

% Medium figure
\newcommand{\mfigg}[3]{
        \begin{figure*}
                \centering
               \hspace*{\fill}\rule{\linewidth}{.5mm}\hspace*{\fill}\vspace{5mm}
                \psfig{figure=#1,width=\linewidth}
                \caption{#3}
                \label{#2}
               \vspace{0mm}\hspace*{\fill}\rule{\linewidth}{.5mm}\hspace*{\fill}
        \end{figure*}
}

% Medium-large figure
\newcommand{\mlfigg}[3]{
        \begin{figure*}[htb]
                \centering
               \hspace*{\fill}\rule{\linewidth}{.5mm}\hspace*{\fill}\vspace{5mm}
                \psfig{figure=#1,height=3.25in}
                \caption{#3}
                \label{#2}
               \vspace{5mm}\hspace*{\fill}\rule{\linewidth}{.5mm}\hspace*{\fill}
        \end{figure*}
}

% Large figure
\newcommand{\lfigg}[3]{
        \begin{figure*}[p]
                \centering
               \hspace*{\fill}\rule{\linewidth}{.5mm}\hspace*{\fill}\vspace{5mm}
                \psfig{figure=#1,height=5in}
                \caption{#3}
                \label{#2}
               \vspace{5mm}\hspace*{\fill}\rule{\linewidth}{.5mm}\hspace*{\fill}
        \end{figure*}
}

% Variable size figure
\newcommand{\vfigg}[4]{
        \begin{figure*}[htb]
                \centering
               \hspace*{\fill}\rule{\linewidth}{.5mm}\hspace*{\fill}\vspace{5mm}
                \psfig{figure=#1,#2}
                \caption{#4}
                \label{#3}
               \vspace{5mm}\hspace*{\fill}\rule{\linewidth}{.5mm}\hspace*{\fill}
        \end{figure*}
}

\newcommand{\vfig}[4]{
        \begin{figure}[ltb]
                \centering
               \hspace*{\fill}\rule{\linewidth}{.5mm}\hspace*{\fill}\vspace{1mm}
                \psfig{figure=#1,#2}
                \caption{#4}
                \label{#3}
               \vspace{1mm}\hspace*{\fill}\rule{\linewidth}{.5mm}\hspace*{\fill}
        \end{figure}
}

\newcommand{\vnlfig}[4]{
        \begin{figure}[htb]
                \centering
               \hspace*{\fill}\rule{\linewidth}{0mm}\hspace*{\fill}\vspace{5mm}
                \psfig{figure=#1,#2}
                \caption{#4}
                \label{#3}
               \vspace{0mm}\hspace*{\fill}\rule{\linewidth}{0mm}\hspace*{\fill}
        \end{figure}
}

\newcommand{\dblvfig}[6]{
        \begin{figure}[htb]
                \centering
                \hspace*{\fill}\rule{\linewidth}{0mm}\hspace*{\fill}\vspace{0.5mm}
                \psfig{figure=#1,#2}
	        \hspace{1in}
                \psfig{figure=#3,#4}
                \caption{#6}
                \label{#5}
               \vspace{2mm}\hspace*{\fill}\rule{\linewidth}{0mm}\hspace*{\fill}
        \end{figure}
}
\newc{\myspacing}{
        \let\oldtextheight=\textheight
        \let\oldtextwidth=\textwidth

        \let\oldtopmargin=\topmargin
        \let\oldheadheight=\headheight
        \let\oldfootheight=\footheight
        \let\oldheadsep=\headsep
        \let\oldoddsidemargin=\oddsidemargin


        \textheight 8.5in
        \textwidth 6in

        \topmargin 0in
        \headheight 0in
        \footheight 1.5in
        \headsep 0in
        \oddsidemargin 0in

}

\newc{\oldspacing}{
        \let\textheight=\oldtextheight 
        \let\textwidth=\oldtextwidth

        \let\topmargin=\oldtopmargin 
        \let\headheight=\oldheadheight 
        \let\footheight=\oldfootheight
        \let\headsep=\oldheadsep
        \let\oddsidemargin=\oldoddsidemargin
}
% Local Variables: 
% mode: latex
% TeX-master: t
% End: 


\begin{document}


\newcounter{listcount}
\newcounter{sublistcount}


\handout{P4}{November 13, 2006}{Instructor: Profs. Feamster/Gray}
{College of Computing, Georgia Tech}{Assignment 4: Conceptual Transfer}

{\em This assignment is due on} {\bf Wednesday, November 22, 2006 at 11:59
  p.m.}  


\section{Purpose of this Assignment}

In Assignment 2, you began exploring a 
way of coming up with new ideas, by
combining papers from disparate and sometimes seemingly unrelated
fields.  This approach operationalizes what Prof. Lipton referred to in
his lecture as working ``at the gap'' between two areas.

In assignment 2 we (deliberately) didn't specify exactly how the
combining was to be done.  In this assignment you'll explore a
powerful variant of this: applying a {\em concept} or tool from one
area to another to either
\begin{enumerate}
\itemsep=-1pt
\item create an {\it analogous} concept in the second area
\item illuminate an existing {\it open problem} in the second area
\end{enumerate}
This is how many ideas which have revolutionized fields came about.
This assignment is intended to provide practice in generating such
``deep thoughts''.  Such conceptual transfer may lead to novel insight
or novel methods, or both.

\section{Some Examples}
An example of a deep analogy between fields
can be seen in learning theory.  The notion of sample complexity, or
the order of growth of the error in approximating a certain kind of
function as a function of the number of data, is analogous to the
concept of runtime complexity, or the order of growth of the runtime
of an algorithm as a function of the size of the input (or number of
data).  This idea leads naturally to interesting analogs of 
concepts such as complexity classes, lower bounds on complexity, etc.
Such concepts have provided one kind of framework for thinking about
how to compare different machine learning algorithms.

An example of applying a concept from one area to help understand a
problem in another is the idea of Frank Kelly, who applied the theory
of markets and economics to help explain how the Internet's congestion
control algorithm induces an equilibrium such that each host gets a
fair share of bandwidth, and such that the capacity of paths are
utilized.  This work did not result in new protocols being deployed,
but it did help network protocol designers gain a much deeper
understanding of why their protocols worked.

\section{Your Task}

Every group has posted its breakdown of the fundamental concepts in 
its area, as well as some open problems in that area.  
Your task is to come up with an idea in one of the two ways
described:
\begin{enumerate}
\itemsep=-1pt
\item adapt a concept from one area to 
create an {\it analogous} concept in a different area
\item adapt a concept from one area to 
illuminate an existing {\it open problem} in a different area
\end{enumerate}
The first and second areas can be any of the group areas, not
necessarily your own.  You may deviate from the particular concepts
and open problems listed by the various groups if that see fit,
keeping in line with our usual ground rules for generating ideas in
this class - as long as you are keeping in the general spirit of the
assignment setup, always feel free to break the rules however you need to 
in order to make a good idea.

\begin{enumerate}
\itemsep=-1pt
\item Each individual is responsible for one idea, and should 
      turn in a one-page writeup.
\item After each individual has selected some idea, each group should
  meet to agree on one idea, which may be a selection of one of the member's
  ideas or some combination of members' ideas, or a jointly created idea.  
  The group should submit a
  one-page writeup, which in addition to arguing for the idea, should 
  demonstrate input from the group as a whole.  For example, if one member's
  idea was selected, there should be an 
  explanation of why the group chose that idea versus the others.
\end{enumerate}

As usual, the one-page writeups, or white papers, should try to argue for:
\begin{itemize}
\itemsep=-1pt
\item the potential impact of this idea on the field
\item the novelty of the idea - this implies a {\it literature survey} 
in which you
place your idea in context with previous work in the field
\end{itemize}

This is the final idea you'll create in this class.  All the ideas
you've created will go into a pool which you will all review in the
next assigment.  The review criteria are threefold: how well you
argued for your idea's impact, how well you argued for your idea's
novelty, and how clearly you write your one-page white paper.  The
paper with the top reviews will win a c.v.-worthy prize.  This is your
last shot so give it a good go.

{\bf This is perhaps the hardest assignment in the class.  It will
require some deep thinking!}  Don't be intimidated - try.
Furthermore, really try to create a truly good idea.  You might invent
your thesis topic during this exercise.

{\em There will be no presentation for this assignment.}

\end{document}
