\documentclass[11pt]{article}

\usepackage{epsf}
\usepackage{epsfig}
\usepackage{url}
\usepackage{../templates/hw}

\newcommand{\newc}{\newcommand}

\newc{\code}[1]{{\tt #1}}
\newc{\func}[1]{{\em #1\/}}

\newc{\be}{\begin{enumerate}}
\newc{\ee}{\end{enumerate}}

\newc{\bi}{\begin{itemize}}
\newc{\ei}{\end{itemize}}

\newc{\bd}{\begin{description}}
\newc{\ed}{\end{description}}

\newc{\ov}[1]{$\overline{#1}$}
\newc{\instr}{\tt}

\newc{\doublespace}{\renewcommand{\baselinestretch}{1.5}}

\newcommand{\figref}[1]{Figure~\ref{#1}}
\newcommand{\tref}[1]{Table~\ref{#1}}

% Captioned table
\newc{\tbl}[3]{
        \begin{table}[htb]
                \centering
                #1
                \caption{#3}
                \label{#2}
        \end{table}
}

% Input a table.
\newcommand{\dblfig}[3]{
        \begin{figure}[htb]
		\centering
                \input{#1}
                \caption{#3}
                \label{#2}
        \end{figure}
}

\newcommand{\ddblfig}[4]{
        \begin{figure}[htb]
		\hspace{-0.1in}
                \psfig{figure=#1,width=0.45\textwidth}
                \caption{#3}
                \label{#2}
        \end{figure}
}

% Figure with no caption
\newcommand{\nofig}[2]{
        \begin{figure}[htb]
                \centering
                \psfig{figure=#1}
                \label{#2}
        \end{figure}
}

% Whole page figure
\newcommand{\schfig}[3]{
        \begin{figure}[p]
                \centering
                \psfig{figure=#1,height=7in}
                \caption{#3}
                \label{#2}
        \end{figure}
}

% Small figure
\newcommand{\sfig}[3]{
        \begin{figure}[ltb]
                \centering
               \hspace*{\fill}\rule{\linewidth}{.5mm}\hspace*{\fill}\vspace{3mm}
                \psfig{figure=#1,width=0.4\textwidth}
                \caption{#3}
                \label{#2}
               \vspace{3mm}\hspace*{\fill}\rule{\linewidth}{.5mm}\hspace*{\fill}
        \end{figure}
}

% Medium figure
\newcommand{\mfig}[3]{
        \begin{figure}[ltb]
		\centering
               \hspace*{\fill}\rule{\linewidth}{.5mm}\hspace*{\fill}\vspace{1mm}
                \psfig{figure=#1,height=2.5in}
                \caption{#3}
                \label{#2}
               \vspace{0mm}\hspace*{\fill}\rule{\linewidth}{.5mm}\hspace*{\fill}
        \end{figure}
}

\newcommand{\widefig}[4]{
        \begin{figure*}[htb]
                \centering
               \hspace*{\fill}\rule{\linewidth}{.5mm}\hspace*{\fill}\vspace{5mm}
                \psfig{figure=#1,width=#3}
                \caption{#4}
                \label{#2}
               \vspace{5mm}\hspace*{\fill}\rule{\linewidth}{.5mm}\hspace*{\fill}
        \end{figure*}
}

\newcommand{\mcfig}[4]{
        \begin{figure}[htbp]
                \centering
               \hspace*{\fill}\rule{\linewidth}{.5mm}\hspace*{\fill}\vspace{5mm}
                \psfig{figure=#1,width=#3}
                \caption{#4}
                \label{#2}
               \vspace{5mm}\hspace*{\fill}\rule{\linewidth}{.5mm}\hspace*{\fill}
        \end{figure}
}

\newcommand{\docfig}[3]{
        \begin{figure}[htbp]
               \hspace*{\fill}\rule{\linewidth}{.5mm}\hspace*{\fill}\vspace{5mm}
                \centering
                \psfig{figure=#1,width=#3}
                \label{#2}
               \vspace{5mm}\hspace*{\fill}\rule{\linewidth}{.5mm}\hspace*{\fill}
        \end{figure}
}

% Medium-large figure
\newcommand{\mlfig}[3]{
        \begin{figure}[htb]
                \centering
               \hspace*{\fill}\rule{\linewidth}{.5mm}\hspace*{\fill}\vspace{5mm}
                \psfig{figure=#1,height=3.25in}
                \caption{#3}
                \label{#2}
               \vspace{5mm}\hspace*{\fill}\rule{\linewidth}{.5mm}\hspace*{\fill}
        \end{figure}
}

% Large figure
\newcommand{\lfig}[3]{
        \begin{figure}[p]
                \centering
               \hspace*{\fill}\rule{\linewidth}{.5mm}\hspace*{\fill}\vspace{5mm}
                \psfig{figure=#1,height=5in}
                \caption{#3}
                \label{#2}
               \vspace{5mm}\hspace*{\fill}\rule{\linewidth}{.5mm}\hspace*{\fill}
        \end{figure}
}

% 'gg' figures are the double column versions of the 'g' figures above.
\newcommand{\sfigg}[3]{
        \begin{figure*}[htb]
                \centering
               \hspace*{\fill}\rule{\linewidth}{.5mm}\hspace*{\fill}\vspace{5mm}
                \psfig{figure=#1,height=1.5in}
                \caption{#3}
                \label{#2}
               \vspace{5mm}\hspace*{\fill}\rule{\linewidth}{.5mm}\hspace*{\fill}
        \end{figure*}
}

% Medium figure
\newcommand{\mfigg}[3]{
        \begin{figure*}
                \centering
               \hspace*{\fill}\rule{\linewidth}{.5mm}\hspace*{\fill}\vspace{5mm}
                \psfig{figure=#1,width=\linewidth}
                \caption{#3}
                \label{#2}
               \vspace{0mm}\hspace*{\fill}\rule{\linewidth}{.5mm}\hspace*{\fill}
        \end{figure*}
}

% Medium-large figure
\newcommand{\mlfigg}[3]{
        \begin{figure*}[htb]
                \centering
               \hspace*{\fill}\rule{\linewidth}{.5mm}\hspace*{\fill}\vspace{5mm}
                \psfig{figure=#1,height=3.25in}
                \caption{#3}
                \label{#2}
               \vspace{5mm}\hspace*{\fill}\rule{\linewidth}{.5mm}\hspace*{\fill}
        \end{figure*}
}

% Large figure
\newcommand{\lfigg}[3]{
        \begin{figure*}[p]
                \centering
               \hspace*{\fill}\rule{\linewidth}{.5mm}\hspace*{\fill}\vspace{5mm}
                \psfig{figure=#1,height=5in}
                \caption{#3}
                \label{#2}
               \vspace{5mm}\hspace*{\fill}\rule{\linewidth}{.5mm}\hspace*{\fill}
        \end{figure*}
}

% Variable size figure
\newcommand{\vfigg}[4]{
        \begin{figure*}[htb]
                \centering
               \hspace*{\fill}\rule{\linewidth}{.5mm}\hspace*{\fill}\vspace{5mm}
                \psfig{figure=#1,#2}
                \caption{#4}
                \label{#3}
               \vspace{5mm}\hspace*{\fill}\rule{\linewidth}{.5mm}\hspace*{\fill}
        \end{figure*}
}

\newcommand{\vfig}[4]{
        \begin{figure}[ltb]
                \centering
               \hspace*{\fill}\rule{\linewidth}{.5mm}\hspace*{\fill}\vspace{1mm}
                \psfig{figure=#1,#2}
                \caption{#4}
                \label{#3}
               \vspace{1mm}\hspace*{\fill}\rule{\linewidth}{.5mm}\hspace*{\fill}
        \end{figure}
}

\newcommand{\vnlfig}[4]{
        \begin{figure}[htb]
                \centering
               \hspace*{\fill}\rule{\linewidth}{0mm}\hspace*{\fill}\vspace{5mm}
                \psfig{figure=#1,#2}
                \caption{#4}
                \label{#3}
               \vspace{0mm}\hspace*{\fill}\rule{\linewidth}{0mm}\hspace*{\fill}
        \end{figure}
}

\newcommand{\dblvfig}[6]{
        \begin{figure}[htb]
                \centering
                \hspace*{\fill}\rule{\linewidth}{0mm}\hspace*{\fill}\vspace{0.5mm}
                \psfig{figure=#1,#2}
	        \hspace{1in}
                \psfig{figure=#3,#4}
                \caption{#6}
                \label{#5}
               \vspace{2mm}\hspace*{\fill}\rule{\linewidth}{0mm}\hspace*{\fill}
        \end{figure}
}
\newc{\myspacing}{
        \let\oldtextheight=\textheight
        \let\oldtextwidth=\textwidth

        \let\oldtopmargin=\topmargin
        \let\oldheadheight=\headheight
        \let\oldfootheight=\footheight
        \let\oldheadsep=\headsep
        \let\oldoddsidemargin=\oddsidemargin


        \textheight 8.5in
        \textwidth 6in

        \topmargin 0in
        \headheight 0in
        \footheight 1.5in
        \headsep 0in
        \oddsidemargin 0in

}

\newc{\oldspacing}{
        \let\textheight=\oldtextheight 
        \let\textwidth=\oldtextwidth

        \let\topmargin=\oldtopmargin 
        \let\headheight=\oldheadheight 
        \let\footheight=\oldfootheight
        \let\headsep=\oldheadsep
        \let\oddsidemargin=\oldoddsidemargin
}
% Local Variables: 
% mode: latex
% TeX-master: t
% End: 


\begin{document}


\newcounter{listcount}
\newcounter{sublistcount}


\handout{A1}{September 14, 2007}{Instructor: Profs. Feamster/Gray}
{College of Computing, Georgia Tech}{Assignment 1: Recognizing Good Ideas}

{\em Complete this assignment individually. \\  This
  assignment is due on} {\bf Friday, October 5, 2007 by 11:59p.m. to
  the TAs.}

\section{Purpose of this Assignment}

This assignment is the first in a series of assignments where you will
take the steps needed to do great research.  In this assignment, you
will perform one of the most common tasks that we as researchers perform
to find a research problem: you will read the most recent proceedings
from the top conference in your area, select two papers that you think
represent ``good research'', and present one idea (for each paper) on
how to extend that research (sometimes hints are presented in the papers
themselves!).


\section{Problem}


One of the first tasks that a researcher must do---not to mention one of
the most challenging ones---is finding the right problem or problems to
work on.  What makes a great research problem?  

First, it's important to realize that there's not necessarily a
``right'' answer to this question, and that research is often a matter
of taste.  Second, it's often difficult to tell whether a research
problem is truly great at first cut (see today's reading for examples
where reviewers ``got it wrong'' the first time around).

Nevertheless, while there are no hard and fast criteria, great research
problems tend to share many of the following characteristics.  Here are
some of our thoughts on what makes good research; when perusing
conference proceedings, it may help you to keep some of the following
criteria in mind when selecting papers:
\begin{itemize}
\item {\em The problem is difficult.}  Life is too short to solve easy
  problems. 
\item {\em Solving the problem creates new knowledge.} Recognizing the
  difference between research and ``a simple matter of engineering'' is
  important.  Many problems are difficult, but if an army of programmers
  could solve the problem with what we know today, it's probably not
  research.
\item {\em The problem changes conventional thinking.} While there is
  value in confirming conventional wisdom, great research typically
  offers surprising results, creates a new way of thinking about or
  approaching problems.
\item {\em The problem has a solution.} Not every paper needs to solve a
  problem completely (in fact, you'll go looking for future work as part
  of this assignment).  Still, while posing a good question is
  paramount, the solution or finding presented in a paper should
  represent a significant advance beyond the previous body of
  knowledge.  As a Ph.D. student, since you ultimately need to graduate,
  it's good to pick problems that have some solution!
\end{itemize}
\noindent
Needless to say, it's difficult when first starting out on a problem do
determine whether it has these characteristics, but it is often patently
clear when a problem does {\em not} have them.

\section{Task}


{\bf Read conference proceedings.}
Each research group should select two papers from their respective
conference proceedings.  Please use the following conference proceedings
as references:

\begin{itemize}
\itemsep=-1.5pt
\item Architecture: {\em International Conference on
Architectural Support for Programming Languages and Operating Systems
(ASPLOS)}, March 2007
\item Artificial Intelligence: {\em International Joint Conference on
  Artificial Intelligence}, January 2007
\item Cognitive Science: {\em Cognitive Science Society Annual Meeting
  (CogSci)},  August 2007
\item Computational Science and Engineering: {\em SIAM Annual Meeting},
  June 2007
\item Cryptography: {\em CRYPTO}, August 2007
\item Databases: {\em ACM SIGMOD/PODS}, June 2007
\item Graphics: {\em ACM SIGGRAPH}, August 2007
\item HCI: {\em CHI}, April 2007
\item Learning Science and Technology: {\em International Conference of
  the Learning Sciences}, June 2006
\item Machine Learning: {\em Neural Information Processing Systems
  (NIPS)}, December 2006
\item Networking: {\em Proceedings of ACM SIGCOMM}, September 2007
\item Programming Languanges: {\em ACM Programming Language Design and
  Implementation}, June 2007
\item Robotics: {\em Robotics Science and Systems}, June 2007
\item Security: {\em IEEE Symposium on Security and Privacy}, May 2007
\item Software Engineering: {\em ACM SIGSOFT Symposium on the
  Foundations of Software Engineering (FSE)}, September 2007
\item Systems: {\em ACM Symposium on Operating Systems Principles},
  October 2007
\item Theory: {\em ACM Symposium on Theory of Computing}, June 2007
\item Vision: {\em Computer Vision and Pattern Recognition (CVPR)}, June
  2007
\end{itemize}

\noindent
{\bf Select and defend papers.}
After selecting your two papers, write a one-page summary of each paper
that provides (1) A short summary of the paper; (2) Why you think it
satisfies the criteria for good research; (3) A short problem statement
for a possible follow-on research problem.



\end{document}
