\documentclass[11pt]{article}

\usepackage{epsf}
\usepackage{epsfig}
\usepackage{url}
\usepackage{../templates/hw}

\newcommand{\newc}{\newcommand}

\newc{\code}[1]{{\tt #1}}
\newc{\func}[1]{{\em #1\/}}

\newc{\be}{\begin{enumerate}}
\newc{\ee}{\end{enumerate}}

\newc{\bi}{\begin{itemize}}
\newc{\ei}{\end{itemize}}

\newc{\bd}{\begin{description}}
\newc{\ed}{\end{description}}

\newc{\ov}[1]{$\overline{#1}$}
\newc{\instr}{\tt}

\newc{\doublespace}{\renewcommand{\baselinestretch}{1.5}}

\newcommand{\figref}[1]{Figure~\ref{#1}}
\newcommand{\tref}[1]{Table~\ref{#1}}

% Captioned table
\newc{\tbl}[3]{
        \begin{table}[htb]
                \centering
                #1
                \caption{#3}
                \label{#2}
        \end{table}
}

% Input a table.
\newcommand{\dblfig}[3]{
        \begin{figure}[htb]
		\centering
                \input{#1}
                \caption{#3}
                \label{#2}
        \end{figure}
}

\newcommand{\ddblfig}[4]{
        \begin{figure}[htb]
		\hspace{-0.1in}
                \psfig{figure=#1,width=0.45\textwidth}
                \caption{#3}
                \label{#2}
        \end{figure}
}

% Figure with no caption
\newcommand{\nofig}[2]{
        \begin{figure}[htb]
                \centering
                \psfig{figure=#1}
                \label{#2}
        \end{figure}
}

% Whole page figure
\newcommand{\schfig}[3]{
        \begin{figure}[p]
                \centering
                \psfig{figure=#1,height=7in}
                \caption{#3}
                \label{#2}
        \end{figure}
}

% Small figure
\newcommand{\sfig}[3]{
        \begin{figure}[ltb]
                \centering
               \hspace*{\fill}\rule{\linewidth}{.5mm}\hspace*{\fill}\vspace{3mm}
                \psfig{figure=#1,width=0.4\textwidth}
                \caption{#3}
                \label{#2}
               \vspace{3mm}\hspace*{\fill}\rule{\linewidth}{.5mm}\hspace*{\fill}
        \end{figure}
}

% Medium figure
\newcommand{\mfig}[3]{
        \begin{figure}[ltb]
		\centering
               \hspace*{\fill}\rule{\linewidth}{.5mm}\hspace*{\fill}\vspace{1mm}
                \psfig{figure=#1,height=2.5in}
                \caption{#3}
                \label{#2}
               \vspace{0mm}\hspace*{\fill}\rule{\linewidth}{.5mm}\hspace*{\fill}
        \end{figure}
}

\newcommand{\widefig}[4]{
        \begin{figure*}[htb]
                \centering
               \hspace*{\fill}\rule{\linewidth}{.5mm}\hspace*{\fill}\vspace{5mm}
                \psfig{figure=#1,width=#3}
                \caption{#4}
                \label{#2}
               \vspace{5mm}\hspace*{\fill}\rule{\linewidth}{.5mm}\hspace*{\fill}
        \end{figure*}
}

\newcommand{\mcfig}[4]{
        \begin{figure}[htbp]
                \centering
               \hspace*{\fill}\rule{\linewidth}{.5mm}\hspace*{\fill}\vspace{5mm}
                \psfig{figure=#1,width=#3}
                \caption{#4}
                \label{#2}
               \vspace{5mm}\hspace*{\fill}\rule{\linewidth}{.5mm}\hspace*{\fill}
        \end{figure}
}

\newcommand{\docfig}[3]{
        \begin{figure}[htbp]
               \hspace*{\fill}\rule{\linewidth}{.5mm}\hspace*{\fill}\vspace{5mm}
                \centering
                \psfig{figure=#1,width=#3}
                \label{#2}
               \vspace{5mm}\hspace*{\fill}\rule{\linewidth}{.5mm}\hspace*{\fill}
        \end{figure}
}

% Medium-large figure
\newcommand{\mlfig}[3]{
        \begin{figure}[htb]
                \centering
               \hspace*{\fill}\rule{\linewidth}{.5mm}\hspace*{\fill}\vspace{5mm}
                \psfig{figure=#1,height=3.25in}
                \caption{#3}
                \label{#2}
               \vspace{5mm}\hspace*{\fill}\rule{\linewidth}{.5mm}\hspace*{\fill}
        \end{figure}
}

% Large figure
\newcommand{\lfig}[3]{
        \begin{figure}[p]
                \centering
               \hspace*{\fill}\rule{\linewidth}{.5mm}\hspace*{\fill}\vspace{5mm}
                \psfig{figure=#1,height=5in}
                \caption{#3}
                \label{#2}
               \vspace{5mm}\hspace*{\fill}\rule{\linewidth}{.5mm}\hspace*{\fill}
        \end{figure}
}

% 'gg' figures are the double column versions of the 'g' figures above.
\newcommand{\sfigg}[3]{
        \begin{figure*}[htb]
                \centering
               \hspace*{\fill}\rule{\linewidth}{.5mm}\hspace*{\fill}\vspace{5mm}
                \psfig{figure=#1,height=1.5in}
                \caption{#3}
                \label{#2}
               \vspace{5mm}\hspace*{\fill}\rule{\linewidth}{.5mm}\hspace*{\fill}
        \end{figure*}
}

% Medium figure
\newcommand{\mfigg}[3]{
        \begin{figure*}
                \centering
               \hspace*{\fill}\rule{\linewidth}{.5mm}\hspace*{\fill}\vspace{5mm}
                \psfig{figure=#1,width=\linewidth}
                \caption{#3}
                \label{#2}
               \vspace{0mm}\hspace*{\fill}\rule{\linewidth}{.5mm}\hspace*{\fill}
        \end{figure*}
}

% Medium-large figure
\newcommand{\mlfigg}[3]{
        \begin{figure*}[htb]
                \centering
               \hspace*{\fill}\rule{\linewidth}{.5mm}\hspace*{\fill}\vspace{5mm}
                \psfig{figure=#1,height=3.25in}
                \caption{#3}
                \label{#2}
               \vspace{5mm}\hspace*{\fill}\rule{\linewidth}{.5mm}\hspace*{\fill}
        \end{figure*}
}

% Large figure
\newcommand{\lfigg}[3]{
        \begin{figure*}[p]
                \centering
               \hspace*{\fill}\rule{\linewidth}{.5mm}\hspace*{\fill}\vspace{5mm}
                \psfig{figure=#1,height=5in}
                \caption{#3}
                \label{#2}
               \vspace{5mm}\hspace*{\fill}\rule{\linewidth}{.5mm}\hspace*{\fill}
        \end{figure*}
}

% Variable size figure
\newcommand{\vfigg}[4]{
        \begin{figure*}[htb]
                \centering
               \hspace*{\fill}\rule{\linewidth}{.5mm}\hspace*{\fill}\vspace{5mm}
                \psfig{figure=#1,#2}
                \caption{#4}
                \label{#3}
               \vspace{5mm}\hspace*{\fill}\rule{\linewidth}{.5mm}\hspace*{\fill}
        \end{figure*}
}

\newcommand{\vfig}[4]{
        \begin{figure}[ltb]
                \centering
               \hspace*{\fill}\rule{\linewidth}{.5mm}\hspace*{\fill}\vspace{1mm}
                \psfig{figure=#1,#2}
                \caption{#4}
                \label{#3}
               \vspace{1mm}\hspace*{\fill}\rule{\linewidth}{.5mm}\hspace*{\fill}
        \end{figure}
}

\newcommand{\vnlfig}[4]{
        \begin{figure}[htb]
                \centering
               \hspace*{\fill}\rule{\linewidth}{0mm}\hspace*{\fill}\vspace{5mm}
                \psfig{figure=#1,#2}
                \caption{#4}
                \label{#3}
               \vspace{0mm}\hspace*{\fill}\rule{\linewidth}{0mm}\hspace*{\fill}
        \end{figure}
}

\newcommand{\dblvfig}[6]{
        \begin{figure}[htb]
                \centering
                \hspace*{\fill}\rule{\linewidth}{0mm}\hspace*{\fill}\vspace{0.5mm}
                \psfig{figure=#1,#2}
	        \hspace{1in}
                \psfig{figure=#3,#4}
                \caption{#6}
                \label{#5}
               \vspace{2mm}\hspace*{\fill}\rule{\linewidth}{0mm}\hspace*{\fill}
        \end{figure}
}
\newc{\myspacing}{
        \let\oldtextheight=\textheight
        \let\oldtextwidth=\textwidth

        \let\oldtopmargin=\topmargin
        \let\oldheadheight=\headheight
        \let\oldfootheight=\footheight
        \let\oldheadsep=\headsep
        \let\oldoddsidemargin=\oddsidemargin


        \textheight 8.5in
        \textwidth 6in

        \topmargin 0in
        \headheight 0in
        \footheight 1.5in
        \headsep 0in
        \oddsidemargin 0in

}

\newc{\oldspacing}{
        \let\textheight=\oldtextheight 
        \let\textwidth=\oldtextwidth

        \let\topmargin=\oldtopmargin 
        \let\headheight=\oldheadheight 
        \let\footheight=\oldfootheight
        \let\headsep=\oldheadsep
        \let\oddsidemargin=\oldoddsidemargin
}
% Local Variables: 
% mode: latex
% TeX-master: t
% End: 


\begin{document}


\newcounter{listcount}
\newcounter{sublistcount}


\handout{A3}{November 21, 2007}{Instructor: Profs. Feamster/Gray}
{College of Computing, Georgia Tech}{Assignment 3: Mock PC}

{\em Complete this assignment individually. \\  This
  assignment is due on} {\bf Monday December 3, 2007 by 11:59p.m. to
  the TAs.}

\section{Purpose of this Assignment}



As you probably know by now, research continually undergoes what is
known as peer review, whereby other people in your discipline, and in
the broader field of computer science, are called upon to evaluate your
work.  

Although the process of peer review is not perfect (in fact, as you can
read in one of the references below, it is often very imperfect!), it is
one of the most important (and most common) ways you will be evaluated
as a researcher.  As the term ``peer'' implies, you will be called upon
to review the work of other researchers.  In this assignment, you will
write a review of each other's research project proposals from
Assignment~2.  The purpose of this assignment is to give you practice
writing reviews of each other's work.

In practice, you will often be given papers, proposals and projects to
review whose subject matter falls outside your immediate area of
expertise.  When this happens, you are still expected to give as
thorough a review as possible but indicate that you are not an expert in
the area (many review forms ask for a self-assessment of expertise).
This assignment will require you to read some project proposals outside
of your immediate area, as well.  Do your best!  (A good reviewer will
often go beyond reading the actual paper or proposal itself and
familiarize him or herself with the related subject material.)

\section{Problem}



You may find some of the following resources helpful:

\begin{itemize}
\itemsep=-1pt
\item {\em We are Sorry to Inform You...} --- \\
  \url{http://ieeexplore.ieee.org/xpl/freeabs_all.jsp?arnumber=1556500}
\item {\em An Evaluation of the Ninth SOSP Submissions} --- \\
  \url{http://www.cs.umbc.edu/cra/etw98/writing-papers.pdf}
\end{itemize}

\section{Task}


The TAs will post the assignments of papers to groups shortly; you will
meet in a group of three students that is different from your original
group.  

{\bf Write project reviews.}
For each set of projects that you are asked to review, you
should (1)~rate your level of expertise in the {\em general} subject
area on a score of 1-5; and (2)~rate each project from 1-5 according to
the following categories:

\begin{itemize}
\itemsep=-1pt
\item Novelty: How unique or new is the idea?  Has the work been placed
  in the context of the related previous work?
\item Potential impact/Technical merit: How likely is the idea to
  advance its particular field significantly?  How well thought out are
  the ideas?  Is there an evaluation plan?
\item Clarity/Presentation: Are the ideas clearly presented?  Are there
  spelling or grammatical errors?
\end{itemize}

In addition to providing scores for each of these, you are welcome to
provide specific comments on each of those three categories.  {\em You
  should also provide detailed comments on the writeup}, suggesting
changes to the presentation, approach/technique, etc.

{\bf Select the best paper from your batch.}  
Think of your small group as a kind of ``mini program committee''; the
three of you should come to a consensus about the best project proposal
among those that were given to you.  You should nominate this project
for the best project award and provide a short defense of your selection
(imagine yourself arguing at a meeting trying to defend your selected
project!).



\end{document}
