\section{Introduction}

This paper describes a course we developed and taught for the first
time in Fall 2006, CS 7001, as a revamping of the required
introductory course for new students in our Computer Science Ph.D.
program.  We are teaching in Fall 2007 (in progress)
an improved version of the course.

Previously, the course was a semester-long seminar series
intended to present the research of the faculty for
the purposes of advisor selection, with a requirement for short
exploratory ``mini-projects'' for testing the waters with a few faculty
members.  We expanded its scope to
address a number of observations about current graduate education in
Computer Science:

{\bf The general skills of independent research are not taught
programmatically.}  Though diverse topics and research styles exist
within Computer Science, there are universal skills of research and
independent work which could in principle be taught, but are currently
learned in {\it ad hoc}, ill-defined and haphazard fashion, if at all.
In other words, to our knowledge, there is {\it no existing
pedagogical practice for teaching independent research} which is widely
used.
We believe that some of the consequences of the current culture
leaving Ph.D. students to figure out these ``meta'' aspects of
independent research on their own include unnecessary and avoidable
contributions to:
  \begin{itemize}
\itemsep=-1pt
  \item High attrition, or drop-out rates, in Ph.D. programs.
  \item Long graduation times in Ph.D. programs.
  \item Slow ramp-up time toward productivity in research.
  \item Poor job preparation and marketability.
  \item A tendency to follow advisors blindly rather than blaze new directions.
  \end{itemize}

{\bf Computer Science research has an opportunity to change all
other disciplines, yet this is not taught.}  Much current CS
research is of an incremental ``stovepipe'' or inward-looking
nature, even though Computer Science has a unique ability to
fundamentally transform virtually every other discipline.  Though
acknowledged often, we do not teach Ph.D. students in CS to actively
seek out and forge the necessary connections to other disciplines.
Unfortunately, to our knowledge, there is {\it no existing
pedagogical practice for teaching cross-disciplinary research} which is
widely used.  We believe that this results in unnecessary contributions to:
  \begin{itemize}
\itemsep=-1pt
  \item Researchers in other disciplines effectively develop their own
        computational solutions, with less expertise.
  \item Untapped potential, in terms of impact and quality of research.
  \end{itemize}

{\bf Computer Science has an opportunity for broader appeal,
  inclusion, and societal impact.}  Enrollment in Computer Science
  programs at all levels is affected by perceived societal impact, as
  evidenced by the effect of highly-visible but limited glimpses into
  the impact of CS such as the dot-com economic boom and bust, the
  effect of Google on daily life, and demographic perceptions
  regarding computer scientists.  The reality, that these perceptions
  are limited, needs to be transmitted by the leaders of the field---CS
  Ph.D. students are not taught about the need to change these
  perceptions, inspire, and communicate their impact.  We believe
  this deficiency has resulted in consequences regarding:
  \begin{itemize}
\itemsep=-1pt
  \item Dwindling enrollments in CS Ph.D. programs.
  \item Perception that CS is ``dead'' even though demand is high.
  \item Low enrollments by women and minorities.
  \end{itemize}

To correct many of these problems, we have developed a first-of-its-kind
``introduction to graduate research'' course; this course is now in its
second year.  The primary goal of the course is to immerse students in
an environment where they can {\em immediately} become engaged in
high-impact research; a related secondary goal is to teach students the
necessary skills that enable them to quickly ``ramp up'' their research
careers and to set them on a trajectory for fruitful research careers,
as well as marketability.  We have designed a compendium comprising 8
assignments, a research project, a mini-project, and a series of 33
lectures that is organized into modules that give students the
ambition---and the capability---to excel in a graduate research program.
This paper summarizes the high-level goals of the course, how we achieve
these goals with 5 course modules, and our experiences and lessons
teaching the course over the past two years.  Although it might seem
that the course itself contains nothing that is not ``obvious in
hindsight'', feedback and data collected from the first offering of the
course suggest that the course provides significant benefits.



The rest of the paper is organized as follows.
Section~\ref{sec:components} describes the components of the course,
including a high-level overview of the course syllabus.
Section~\ref{sec:progress} describes the progress we have made in
teaching the course to incoming graduate students and discusses various
challenges we encountered, as well as how we addressed them.
Section~\ref{sec:related} describes related work, and
Section~\ref{sec:conclusion} concludes.
