\section{Evaluation and Progress}\label{sec:progress}

We collected data on the Fall 2006 offering of the course
from: (1)~a student focus group in February 2007 (which was attended by
students who took the class in Fall 2006 as well as students in previous
offerings); (2)~a town hall meeting with students and faculty in June
2007; (3)~course evaluations; and (4)~many conversations with students
and faculty (in person and over email).

\subsection{Observations}

We observed the following phenomena based on the above feedback and
data:
\begin{itemize}
\itemsep=-1pt
\item Student attendance was often poor.
\item Student contribution to group assignments was very uneven.
\item Senior students were quick to criticize the course when logistical
  and organizational wrinkles occurred.
\item Senior students told new ones to selectively attend lectures.
\item Both students and faculty expressed tension between mini-projects
  (which encourage exploration) and term-long projects (which encourage
  depth). 
\item We had trouble recruiting high-profile speakers to speak in an
  introductory course to first-year Ph.D. students.
\end{itemize}

\subsection{Inferences}

Integrating and distilling the above observations led us to the
following conclusions:
\begin{itemize}
\itemsep=-1pt
%\item There was a tendency not to take course seriously; this was partially
%due to the culture of the old course, and partially because of the 'soft
%skills' nature of the course.
\item Many students lack the context and experience to fully appreciate the
importance of the course material, which largely comprises ``soft''
skills and advice contained in lectures, rather than traditional
testable material.
\item Giving a choice between the mini-projects and a single main project
created confusion and also resulted in a failure to meet our original
goals for redesigning the course (i.e., involving students in in-depth
research projects early in their graduate careers {\em as well as}
encouraging exploration).
\item Due to inertia, our radical changes to the course were met with
  skepticism from older students who had taken the previous version.
%, though once students and
%faculty understood what we were trying to do, there seemed to be broad
%support.  Many older students and faculty in fact told us that they wish
%they had taken a course like ours.
\end{itemize}

\subsection{Corrective Measures}

As we are only now teaching the second offering of the course, is not
yet clear what the impact of the changes will be, but we are confident
that the current offering incorporates many of the suggestions offered
by students and faculty.  Specifically, we are taking the following
corrective measures:
\begin{enumerate}
\itemsep=-1pt
\item We now offer a cash prize for the best research ideas, now take
attendance, changed the required course from pass/fail to letter-graded,
and changed the time of the course to late afternoon to encourage
attendance.
\item We now require students to perform both a main project and a
mini-project, as they both serve critical functions: the former
providing an opportunity to explore a problem in-depth, and the latter
allows students to explore other research areas and meet other faculty
in the department.
\item In the current instantiation, all assignments are done individually,
with the exception of the mock program committee.  Unfortunately, this
leaves only one activity involving teamwork, which is unfortunate given
the ever-increasing importance of teams in research.  We regard this as
an open issue, which we intend to address in future instances of the
course.
\item A number of obvious logistical wrinkles regarding organization, clarity
of the assignments, and scheduling of the course time were ironed out in
the current instance of the course.
\item We are coupling external speaker invitations with department-wide
distinguished lectures.
\end{enumerate}

