\section{Course Components}\label{sec:components}

This section describes the five components of our introductory course
to the Ph.D. program:
\begin{itemize}
\itemsep=-1pt
\item {\em Research skills} comprise topics that help students
  recognize, generate, critique and communicate research ideas.
\item {\em Research mechanics} include topics that help students develop
  essential techniques for performing various common tasks in computer
  science research.
\item {\em Skills for independent work} comprise lectures that help
  students develop skills for working independently, including how to stay
  motivated, set goals, and manage time.
\item {\em Career development} is an area students should be be working
  on as early as possible in the Ph.D. program so they can select and
  plan for an appropriate career path.
\item {\em Exemplars and bootstrapping.} The course includes a set of
  lectures to help students become better oriented with computer science
  both within the department and at-large.
\end{itemize}

\subsection{Research Skills}

\begin{table}[t]
\begin{center}
\begin{small}
\begin{tabular}{l|p{2in}}
{\bf Assignment} & {\bf Description} \\ \hline
\multicolumn{2}{c}{{\em Assignments}} \\ \hline
Recognizing good ideas & Read proceedings from top conference in field,
select two ``best'' papers, and provide a research summary and defense
for each. \\
Generating good ideas & Read other students' summaries from first
assignments.  Combine two ideas from first assignment to propose a new
research direction, idea, or solution. \\ 
Critiquing ideas & Read proposals of research ideas from second
assignment, write reviews, and meet in a mock program committee to
select ``best'' proposals. \\
Communicating ideas & Deliver a presentation to the class on a term-long
research project.  \\ \hline
\multicolumn{2}{c}{{\em Mini-Assignments}} \\ \hline
Why Ph.D.? & Write down your goals and expectations for the
Ph.D. program.  Also, summarize the characteristics of the researchers
and research results that you most admire. \\
Time audit & Log how time is spent in 10-minute increments over the
course of a week. \\
Web page & Create a personal research Web page. \\
Elevator pitch & Compose ``elevator pitches'' of various lengths
(30-second, 1-minute,  5-minute) and deliver the 30-second pitch to the
class. 
\end{tabular}
\end{small}
\end{center}
\vspace*{-0.1in}
\caption{Summary of assignments.}
\vspace*{-0.2in}
\label{tab:assignments}
\end{table}

\begin{table}[t]
\begin{center}
\begin{small}
\begin{tabular}{p{2.5in}|r}
{\bf Topic} & {\bf Lectures} \\ \hline
\multicolumn{2}{c}{{\em Research Skills}} \\ \hline
{\bf Background: How research works/Impact} & 1 \\
Recognizing good ideas & 0.5 \\
Generating good ideas & 2 \\
- Problem selection and cross-disciplinary work & \\
- Creativity and idea generation \\
Critiquing ideas & 0.5 \\ 
- Reading and reviewing papers\\
Communicating ideas & 2 \\ 
- How to write a paper \\
- How to give a talk \\ \hline
\multicolumn{2}{c}{{\em Research Mechanics}} \\ \hline
{\bf Background: Research patterns} & 1 \\
Math skills & 1 \\
Data and empirical skills & 1\\
Programming skills & 1 \\
Human-centered research skills & 1 \\ \hline
\multicolumn{2}{c}{{\em Skills for Independent Work}} \\ \hline
{\bf Background: Executing Great Research} & 1 \\
Goal setting & 1 \\
- Why Ph.D.? \\
Motivation & 1 \\
Time management & 1 \\
Information management & 1 \\
Personal development & 2 \\ 
- Student life and social activities & \\
- Graduate school survival skills & \\
\hline
\multicolumn{2}{c}{{\em Career Development}} \\ \hline
Overviews of job opportunities & 3 \\
- Professor life\\
- Industry vs. academia\\
- Commercialization \\
Teaching and TAing & 1 \\
Personal and research promotion (networking, etc.) & 1 \\ \hline
\multicolumn{2}{c}{{\em Exemplars and Bootstrapping}} \\ \hline
Internal speakers (department and area overviews) & 6 \\ 
External speakers & 2 \\
Broader impact & 2 \\
- Computing for social good & \\
- Diversity and women in computing & \\
\end{tabular}
\end{small}
\end{center}
\vspace*{-0.1in}
\caption{Syllabus Overview.}
\vspace*{-0.2in}
\label{tab:syllabus}
\end{table}

The first module, research skills, teaches students how to develop
creativity and critical thinking skills needed to perform research.  We
divide research skills into four main tasks, each of which correspond to
an assignment, as shown in Table~\ref{tab:assignments}.

\vspace*{0.1in}
\noindent
{\bf Recognizing good ideas.}  Helping students recognize good research ideas
is critical for helping them develop taste in research problems, as well
as to help them develop essential skills such as reading papers.  To
help students develop the ability to recognize good research ideas, the
course includes a lecture on reading research papers.  We then ask the
students to put this knowledge into practice in the first assignment,
where they must read the papers from the top conference proceedings in
their field, select two papers from that set of proceedings, summarize
the main ideas from these papers, and defend their choices of these
papers as representing good research problems (e.g., the paper solves a
longstanding open research problem, defines a new field or direction).
This assignment not only helps students develop research taste, but it
also helps them get into the habit of reading research papers early in
their research career.

\vspace*{0.1in}
\noindent
{\bf Generating good ideas.}  Once students are equipped with the the skills
to read conference proceedings and have begun to develop their own
research taste, we help them develop the skills to generate research
ideas.  Of course, there is no formula for creativity, but certain
approaches to research can put students in a position where they are
more likely to have creative thought.  In keeping with our goal to
counter ``stovepipe'' research and foster interdisciplinary thinking, we
encourage students to solve problems that exist at the gap between two
research areas.  To this point, we incorporate several lectures on
``research at the gap'' to help develop cross-disciplinary thinking.

The second assignment encourages students to think about
cross-disciplinary research problems by having them generate a research
idea by combining two ideas selected by students in the first assignment
from different research areas.  The assignment is motivated by the
observation that great progress on problems is often made by applying
ideas or concept from one discipline to a second, seemingly unrelated
problem area.  To save time and survey a broader range of ideas,
students can use the research summaries generated by other students,
but, if they so choose, they can also select their own set of problems.
The output from this assignment is the equivalent of a short research
proposal, comprising the following three aspects: (1)~a concise
description of the problem; (2)~an explanation of why the problem is
important and what practical or broader impacts solving the problem
would have; and (3)~an explanation of why the problem is challenging or
interesting.  This assignment not only helps students gain experience in
creative thinking, but it also gives them valuable experience in
communicating their research ideas and explaining their importance,
skills that are useful for writing papers and, eventually, research
proposals.

\vspace*{0.1in}
\noindent
{\bf Critiquing ideas.}  After learning how to generate research ideas,
the course gives students the opportunity to develop critical thinking
skills by writing reviews of other students' research proposals.  The
course includes lectures on how to critically read and review research
papers and proposals (a skill that is also intimately tied to
recognizing good research ideas and developing taste in research
problems).  The course includes a corresponding assignment to help
students write reviews: each student is given a selection of other
students' research proposals from the second assignment to critically
review.  Based on these reviews, students then form mock program
committees to select the ``best'' research proposals, each of which
receives a cash prize.  This assignment helps students understand
several important aspects of research: First, they develop further
experience critiquing other research ideas.  Second, they gain some
insight into how research ideas are selected by program committees.
This insight into the (sometimes imperfect) review process may provide
some comfort to a student when his or her first paper is rejected.

\vspace*{0.1in}
\noindent
{\bf Communicating ideas.}  Equally important to developing the research
ideas themselves is communicating them clearly to others.  We impress
upon students the importance of communicating research ideas as a
critical step towards having their ideas adopted, applied, or built
upon.  Accordingly, the course has many assignments, such as those
above, that involve developing writing skills.  Additionally, students
have a major writing assignment that is tied to their term-long research
project.  In this assignment, students are asked to write a paper---in
conference-paper format, composition, and style---that summarizes the
term-long research project that they perform throughout the duration of
the term.  In addition to developing writing skills, students also have
a mini-assignment that helps them develop multi-resolution ``elevator
pitch'' talks of various lengths on their research that can be used in
different settings.  These assignments help students develop a
multi-faceted approach to communicating and promoting their research
ideas.



\subsection{Research Mechanics}

Students not only need skills for performing research, they also need to
develop mechanics for performing research tasks that are common across
all disciplines of computer science, particularly for cross-disciplinary
tasks.  We focus on developing mechanics in three areas, devoting a
lecture to each: (1)~math and analytical skills; (2)~programming skills;
and (3)~skills for human-centered research and experimentation.  We
briefly survey the material covered for each of these three topics.

\vspace*{0.1in}
\noindent
{\bf Math skills.} For work involving a mathematical/theoretical
component, many students may only have mathematical knowledge but lack a
number of intuitions which emerge only from research practice.  We will
cover issues such as what constitutes a mathematical "theory", varying
levels of rigor in proofs, various proof approaches and strategies, the
importance of good notation, and idea generation in mathematics.

\vspace*{0.1in}
\noindent
{\bf Data and empirical skills.}  Although some areas of computer
science are more empirical than others, many aspects of research involve
experimental design and data analysis.  Thus, the course includes one
lecture on how to design experiments and draw meaningful conclusions
from experimental data.  We pay close attention to avoiding common
pitfalls with experimental design and data analysis, as well as
clearly presenting experimental results.

\vspace*{0.1in}
\noindent
{\bf Programming skills.}  We devote one lecture to essential
programming skills that are common across computer science research.
This lecture discusses specific research programming skills, such as
version control, rapid prototyping, common design patterns,
documentation for public release of prototypes, and programming skills
for executing reproducible experiments.

\vspace*{0.1in}
\noindent
{\bf Skills for human-centered research.}  Various areas of computer
science ranging from computer networking to human-computer interaction,
require either data that is generated from humans or direct interaction
with human subjects.  To help students prepare for this type of
research, the course includes a lecture on various human-centered
research techniques (e.g., ethnography, discourse analysis), as well as
other logistical issues with human-centered research, such as
institutional review board approval processes.  

%testing human subjects
%... ethnography, discourse analysis

\subsection{Skills for Independent Work}

%Although much research is collaborative in nature, Ph.D. students
%ultimately assume a large amount of responsibility in pushing their own
%research forward.  
Unlike earlier stages of education, Ph.D. research is
largely unstructured and involves setting goals on very long time
horizons.  Accordingly, the course includes lectures and assignments
that help students develop skills to work independently towards
seemingly distant goals.  

\vspace*{0.1in}
\noindent
{\bf Goal setting.}  The course includes lectures on setting goals and
systematically and methodically working towards those goals.  On the
first day of the course, we ask students to answer questions that help
them think about why they have enrolled in a Ph.D. program and what they
hope to achieve by the end of the Ph.D.; we have an accompanying lecture
that discusses the various reasons for obtaining a Ph.D. (e.g., we
discuss the various job options that having a Ph.D. enables).  By
encouraging students to explicitly write down their goals early in the
program, we fend off the potentially dangerous situation where students
enter a Ph.D. program as a default option without a codified set of
goals.  We impress upon students that setting goals early in the program
is important both so that they have something concrete to work towards
and also because long-term goals can help inform priorities and tactical
decisions throughout the program (e.g., how to allocate time to various
research projects, when and how to promote one's research, etc.).

\vspace*{0.1in}
\noindent
{\bf Motivation.} The Ph.D. is a long process with many ups and
downs; sustaining a high level of motivation can be challenging.  As
such, the course includes a lecture on how to stay motivated in the face
of common trials and tribulations of a graduate research career (e.g.,
paper rejections, self-doubt about the direction of one's research,
day-to-day motivation).

\vspace*{0.1in}
\noindent
{\bf Time management.} Incoming Ph.D. students are typically accustomed
to working on well-defined assignments with fixed, near-term deadlines.
In contrast, graduate research typically involves working on loosely
defined problems (at least initially) over significantly longer time
periods.  Without appropriate time management skills, a graduate student
can fall into traps at either of the two extremes: either working all
the time (but not necessarily efficiently) or not working at all until a
deadline is imminent.  Either approach can lead to inefficiency, lack of
productivity, and ultimately lack of sufficient progress, resulting in
attrition or longer graduation times.   To combat these problems, the
course includes a lecture and a mini-assignment on time management.  We
teach students common tips and tricks for managing time and also have
them perform a ``time audit'' assignment, whereby the student records
how he or she spends an entire work week, broken down into ten-minute
increments.   Students are surprised to learn in this assignment that
what might seem like a 12-hour workday is filled with significant
``gaps'' of unproductive activity and interruptions.   

\vspace*{0.1in}
\noindent
{\bf Information management.}  Another difficulty faced by
Ph.D. students is managing the thoughts and ideas that arise when
students read papers, attend talks, meet with their advisors, and so
forth.  We thus include course material managing information, such as
maintaining a research notebook, a research project wiki, and how to
manage email correspondence and notes.  We also introduce tips on how to
have productive advisor meetings, which includes important information
management skills such as coming to the meeting with an agenda, taking
notes on the meeting and sending a post-meeting summary, etc.

\vspace*{0.1in}
\noindent
{\bf Personal development and adjustment.}  Researchers cannot be
productive if they are not happy personally.  Accordingly, we also
include several ``personal development'' aspects in the course.  We
scheduled the class period in the late afternoon to facilitate
transition to community-building social events (e.g., picnics, happy
hours), and incorporated a student panel on living in Atlanta.  We also
incorporated a panel where more senior Ph.D. students shared their
experiences about life in the graduate program.


\subsection{Career Development}

A systematic introduction to employment possibilities early in the
Ph.D. program can help students prepare for their career after graduate
school before it is too late for many options to be viable.  Keeping
certain career options open (e.g., academia) often require building a
solid research reputation over many years.  To help students appreciate
the full range of post-Ph.D. employment possibilities and make
appropriate career choices to help them be marketable when they
graduate, we include course material that explains both job
opportunities and promotion of one's research.

\vspace*{0.1in}
\noindent
{\bf Overviews of job opportunities.}  As previously mentioned, the
second lecture of the course, ``Why a Ph.D.?'' includes a high-level
overview of the job opportunities that either become viable as a result
of having a Ph.D. or are natural consequences of graduate research:
academia, industrial or government research, and entrepreneurship.  We
include specific lectures on life as a professor, the differences
between working in industry vs. academia (including guest speakers from
researchers who have had experience in both areas), and how to
commercialize one's research.  

\vspace*{0.1in}
\noindent
{\bf Personal and research promotion.}  An important factor in a
student's marketability at the time of graduation is how well known the
student and his or her research is in the broader research community.
This oft-overlooked aspect of a student's development as a researcher
can sometimes cost a student opportunities in academic and industry
research positions, where employment opportunities often correlate with
other researchers' familiarity with the student's research work.  To
help students form their research reputations early, the course includes
a lecture on publicizing one's research in the broader community (e.g.,
popular press), as well as lecture material on networking within the
professional community; it also includes mini-assignment where students
must construct a personal research Web page.

\subsection{Exemplars and Bootstrapping}

One of the most difficult aspects of the graduate career is getting
started and, more importantly, appreciating that one's research can, in
fact, have broader impacts.  Thus, a primary goal of the course is to
get students involved in research as soon as possible (i.e.,
immediately) and to appreciate that their work can, in fact, have
broader impacts.  To achieve this goal, we involve students in projects
that incorporate both breadth and depth, and we provide students with
specific examples of computer science research projects that have had
real-world impact.  We also provide basic logistical information to help
students bootstrap at the beginning of their research careers.

\vspace*{0.1in}
\noindent
{\bf Main project and mini-project.}  We require the students to perform
one ``main'' term-long research project (typically with their {\em de
  facto} advisor) and one exploratory mini-project.  The main project
has several benefits: First, it gives students a sense of immediate
accomplishment and defends against students' tendency to feel ``lost''
when they first arrive in the program.  Second, it quickly provides
students with a sense of what it is like to do research; this experience
allows them to quickly discover both the joy and challenge of research,
to appreciate the difference between graduate school and undergraduate
education, to adapt their working skills as needed to be successful, and
to decide whether they like research at all.  The mini-project can be
done with any faculty member in the department and is intended to give
the students an excuse to explore research topics that may be further
afield from their main interest (and, incidentally, might prove useful
in helping them apply cross-disciplinary thinking to their own
research). 

\vspace*{0.1in}
\noindent
{\bf Internal and external speakers.}  The course includes an internal
lecture series to help students familiarize themselves with various
aspects of the department.  Rather than simply providing an overview or
advertisement, however, we asked faculty within the department to
provide a research overview in the form of ``big picture'' research
questions (e.g., What are the five biggest open questions in Computer
Science research? What are current research trends?).  We also include a
lecture from the dean that includes a discussion of broader research
directions, and we have both students and faculty give advice on other
logistical topics, such as how to apply for fellowships.

\vspace*{0.1in}
\noindent
{\bf Broader impact topics.}  Anecdotal evidence suggested that many
Ph.D. students fail to complete their studies because they don't get a
sense that their work has important broad impacts.  To account for this,
we include several lectures in the course on broader impact topics
(e.g., how computing can help people in developing regions, integration
of research with diversity and education).  
