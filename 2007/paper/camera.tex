\begin{abstract}
This paper describes a course we have developed for preparing new Ph.D.
students in computer science for a career in research.  The course is
intended to teach the skills needed for research and independent work,
prepare students psychologically and socially for years lying before
them, and help them find a good Ph.D. topic by providing principles and
examples.  In this course, we emphasize and encourage impact through
cross-disciplinary research and broader societal outreach.  To our
knowledge, the course represents a first-of-its-kind systematic
introduction to a graduate research career.  This paper describes our
high-level goals for this curricular initiative, the structure of the
course (including lecture components and assignments), and the
challenges we faced in developing this course.  As we continue to
develop this course, which is now in its second year, we hope it will
serve as a model ``introduction of Ph.D. research'' course for other
computer science departments.
\end{abstract}
\section{Course Components}\label{sec:components}

In this section, we describe the components of our introductory course
to the Ph.D. program.  We have organized the course into five
components, which we describe in detail in the rest of the section:

\begin{itemize}
\itemsep=-1pt
\item {\em Research skills} comprise topics that help students
  recognize, generate, critique and communicate research ideas.
\item {\em Research mechanics} include topics that help students develop
  essential techniques for performing various common tasks in computer
  science research.
\item {\em Skills for independent work} comprise lectures that help
  students develop skills for working independently, including how to stay
  motivated, set goals, and manage time.
\item {\em Career development} is an area students should be be working
  on as early as possible in the Ph.D. program so they can select and
  plan for an appropriate career path.
\item {\em Exemplars and bootstrapping.} The course includes a set of
  lectures to help students become better oriented with computer science
  both within the department and at-large.
\end{itemize}

\subsection{Research Skills}

\begin{table}[t]
\begin{center}
\begin{small}
\begin{tabular}{l|p{2in}}
{\bf Assignment} & {\bf Description} \\ \hline
\multicolumn{2}{c}{{\em Assignments}} \\ \hline
Recognizing good ideas & Read proceedings from top conference in field,
select two ``best'' papers, and provide a research summary and defense
for each. \\
Generating good ideas & Read other students' summaries from first
assignments.  Combine two ideas from first assignment to propose a new
research direction, idea, or solution. \\ 
Critiquing ideas & Read proposals of research ideas from second
assignment, write reviews, and meet in a mock program committee to
select ``best'' proposals. \\
Communicating ideas & Deliver a presentation to the class on a term-long
research project.  \\ \hline
\multicolumn{2}{c}{{\em Mini-Assignments}} \\ \hline
Why Ph.D.? & Write down your goals and expectations for the
Ph.D. program.  Also, summarize the characteristics of the researchers
and research results that you most admire. \\
Time audit & Log how time is spent in 10-minute increments over the
course of a week. \\
Web page & Create a personal research Web page. \\
Elevator pitch & Compose ``elevator pitches'' of various lengths
(30-second, 1-minute,  5-minute) and deliver the 30-second pitch to the
class. 
\end{tabular}
\end{small}
\end{center}
\vspace*{-0.1in}
\caption{Summary of assignments.}
\vspace*{-0.2in}
\label{tab:assignments}
\end{table}

\begin{table}[t]
\begin{center}
\begin{small}
\begin{tabular}{p{2.5in}|r}
{\bf Topic} & {\bf Lectures} \\ \hline
\multicolumn{2}{c}{{\em Research Skills}} \\ \hline
{\bf Background: How research works/Impact} & 1 \\
Recognizing good ideas & 0.5 \\
Generating good ideas & 2 \\
- Problem selection and cross-disciplinary work & \\
- Creativity and idea generation \\
Critiquing ideas & 0.5 \\ 
- Reading and reviewing papers\\
Communicating ideas & 2 \\ 
- How to write a paper \\
- How to give a talk \\ \hline
\multicolumn{2}{c}{{\em Research Mechanics}} \\ \hline
{\bf Background: Research patterns} & 1 \\
Math skills & 1 \\
Data and empirical skills & 1\\
Programming skills & 1 \\
Human-centered research skills & 1 \\ \hline
\multicolumn{2}{c}{{\em Skills for Independent Work}} \\ \hline
{\bf Background: Executing Great Research} & 1 \\
Goal setting & 1 \\
- Why Ph.D.? \\
Motivation & 1 \\
Time management & 1 \\
Information management & 1 \\
Personal development & 2 \\ 
- Student life and social activities & \\
- Graduate school survival skills & \\
\hline
\multicolumn{2}{c}{{\em Career Development}} \\ \hline
Overviews of job opportunities & 3 \\
- Professor life\\
- Industry vs. academia\\
- Commercialization \\
Teaching and TAing & 1 \\
Personal and research promotion (networking, etc.) & 1 \\ \hline
\multicolumn{2}{c}{{\em Exemplars and Bootstrapping}} \\ \hline
Internal speakers (department and area overviews) & 6 \\ 
External speakers & 2 \\
Broader impact & 2 \\
- Computing for social good & \\
- Diversity and women in computing & \\
\end{tabular}
\end{small}
\end{center}
\vspace*{-0.1in}
\caption{Syllabus Overview.}
\vspace*{-0.2in}
\label{tab:syllabus}
\end{table}

The first module, research skills, teaches students how to develop
creativity and critical thinking skills needed to perform research.  We
divide research skills into four main tasks, each of which correspond to
an assignment, as shown in Table~\ref{tab:assignments}.

\vspace*{0.1in}
\noindent
{\bf Recognizing good ideas.}  Helping students recognize good research ideas
is critical for helping them develop taste in research problems, as well
as to help them develop essential skills such as reading papers.  To
help students develop the ability to recognize good research ideas, the
course includes a lecture on reading research papers.  We then ask the
students to put this knowledge into practice in the first assignment,
where they must read the papers from the top conference proceedings in
their field, select two papers from that set of proceedings, summarize
the main ideas from these papers, and defend their choices of these
papers as representing good research problems (e.g., the paper solves a
longstanding open research problem, defines a new field or direction).
This assignment not only helps students develop research taste, but it
also helps them get into the habit of reading research papers early in
their research career.

\vspace*{0.1in}
\noindent
{\bf Generating good ideas.}  Once students are equipped with the the skills
to read conference proceedings and have begun to develop their own
research taste, we help them develop the skills to generate research
ideas.  Of course, there is no formula for creativity, but certain
approaches to research can put students in a position where they are
more likely to have creative thought.  In keeping with our goal to
counter ``stovepipe'' research and foster interdisciplinary thinking, we
encourage students to solve problems that exist at the gap between two
research areas.  To this point, we incorporate several lectures on
``research at the gap'' to help develop cross-disciplinary thinking.

The second assignment encourages students to think about
cross-disciplinary research problems by having them generate a research
idea by combining two ideas selected by students in the first assignment
from different research areas.  The assignment is motivated by the
observation that great progress on problems is often made by applying
ideas or concept from one discipline to a second, seemingly unrelated
problem area.  To save time and survey a broader range of ideas,
students can use the research summaries generated by other students,
but, if they so choose, they can also select their own set of problems.
The output from this assignment is the equivalent of a short research
proposal, comprising the following three aspects: (1)~a concise
description of the problem; (2)~an explanation of why the problem is
important and what practical or broader impacts solving the problem
would have; and (3)~an explanation of why the problem is challenging or
interesting.  This assignment not only helps students gain experience in
creative thinking, but it also gives them valuable experience in
communicating their research ideas and explaining their importance,
skills that are useful for writing papers and, eventually, research
proposals.

\vspace*{0.1in}
\noindent
{\bf Critiquing ideas.}  After learning how to generate research ideas,
the course gives students the opportunity to develop critical thinking
skills by writing reviews of other students' research proposals.  The
course includes lectures on how to critically read and review research
papers and proposals (a skill that is also intimately tied to
recognizing good research ideas and developing taste in research
problems).  The course includes a corresponding assignment to help
students write reviews: each student is given a selection of other
students' research proposals from the second assignment to critically
review.  Based on these reviews, students then form mock program
committees to select the ``best'' research proposals, each of which
receives a cash prize.  This assignment helps students understand
several important aspects of research: First, they develop further
experience critiquing other research ideas.  Second, they gain some
insight into how research ideas are selected by program committees.
This insight into the (sometimes imperfect) review process may provide
some comfort to a student when his or her first paper is rejected.

\vspace*{0.1in}
\noindent
{\bf Communicating ideas.}  Equally important to developing the research
ideas themselves is communicating them clearly to others.  We impress
upon students the importance of communicating research ideas as a
critical step towards having their ideas adopted, applied, or built
upon.  Accordingly, the course has many assignments, such as those
above, that involve developing writing skills.  Additionally, students
have a major writing assignment that is tied to their term-long research
project.  In this assignment, students are asked to write a paper---in
conference-paper format, composition, and style---that summarizes the
term-long research project that they perform throughout the duration of
the term.  In addition to developing writing skills, students also have
a mini-assignment that helps them develop multi-resolution ``elevator
pitch'' talks of various lengths on their research that can be used in
different settings.  This collection of assignments helps students
develop a multi-faceted approach to communicating and promoting their
research ideas.  



\subsection{Research Mechanics}

Students not only need skills for performing research, they also need to
develop mechanics for performing research tasks that are common across
all disciplines of computer science, particularly for cross-disciplinary
tasks.  We focus on developing mechanics in three areas, devoting a
lecture to each: (1)~math and analytical skills; (2)~programming skills;
and (3)~skills for human-centered research and experimentation.  We
briefly survey the material covered for each of these three topics.

\vspace*{0.1in}
\noindent
{\bf Math skills.} For work involving a mathematical/theoretical
component, many students may only have mathematical knowledge but lack a
number of intuitions which emerge only from research practice.  We will
cover issues such as what constitutes a mathematical "theory", varying
levels of rigor in proofs, various proof approaches and strategies, the
importance of good notation, and idea generation in mathematics.

\vspace*{0.1in}
\noindent
{\bf Data and empirical skills.}  Although some areas of computer
science are more empirical than others, many aspects of research involve
experimental design and data analysis.  Thus, the course includes one
lecture on how to design experiments and draw meaningful conclusions
from experimental data.  We pay close attention to avoiding common
pitfalls with experimental design and data analysis, as well as
clearly presenting experimental results.

\vspace*{0.1in}
\noindent
{\bf Programming skills.}  We devote one lecture to essential
programming skills that are common across computer science research.
This lecture discusses specific research programming skills, such as
version control, rapid prototyping, common design patterns,
documentation for public release of prototypes, and programming skills
for executing reproducible experiments.

\vspace*{0.1in}
\noindent
{\bf Skills for human-centered research.}  Various areas of computer
science ranging from computer networking to human-computer interaction,
require either data that is generated from humans or direct interaction
with human subjects.  To help students prepare for this type of
research, the course includes a lecture on various human-centered
research techniques (e.g., ethnography, discourse analysis), as well as
other logistical issues with human-centered research, such as
institutional review board approval processes.  

%testing human subjects
%... ethnography, discourse analysis


\subsection{Skills for Independent Work}

Although much research is collaborative in nature, Ph.D. students
ultimately assume a large amount of responsibility in pushing their own
research forward.  Unlike earlier stages of education, Ph.D. research is
largely unstructured and involves setting goals on very long time
horizons.  Accordingly, the course includes lectures and assignments
that help students develop skills to work independently towards
seemingly distant goals.  

\vspace*{0.1in}
\noindent
{\bf Goal setting.}  The course includes lectures on setting goals and
systematically and methodically working towards those goals.  On the
first day of the course, we ask students to answer questions that help
them think about why they have enrolled in a Ph.D. program and what they
hope to achieve by the end of the Ph.D.; we have an accompanying lecture
that discusses the various reasons for obtaining a Ph.D. (e.g., we
discuss the various job options that having a Ph.D. enables).  By
encouraging students to explicitly write down their goals early in the
program, we fend off the potentially dangerous situation where students
enter a Ph.D. program as a default option without a codified set of
goals.  We impress upon students that setting goals early in the program
is important both so that they have something concrete to work towards
and also because long-term goals can help inform priorities and tactical
decisions throughout the program (e.g., how to allocate time to various
research projects, when and how to promote one's research, etc.).

\vspace*{0.1in}
\noindent
{\bf Motivation.} The Ph.D. is a long process with many ups and
downs; sustaining a high level of motivation can be challenging.  As
such, the course includes a lecture on how to stay motivated in the face
of common trials and tribulations of a graduate research career (e.g.,
paper rejections, self-doubt about the direction of one's research,
day-to-day motivation).

\vspace*{0.1in}
\noindent
{\bf Time management.} Incoming Ph.D. students are typically accustomed
to working on well-defined assignments with fixed, near-term deadlines.
In contrast, graduate research typically involves working on loosely
defined problems (at least initially) over significantly longer time
periods.  Without appropriate time management skills, a graduate student
can fall into traps at either of the two extremes: either working all
the time (but not necessarily efficiently) or not working at all until a
deadline is imminent.  Either approach can lead to inefficiency, lack of
productivity, and ultimately lack of sufficient progress, resulting in
attrition or longer graduation times.   To combat these problems, the
course includes a lecture and a mini-assignment on time management.  We
teach students common tips and tricks for managing time and also have
them perform a ``time audit'' assignment, whereby the student records
how he or she spends an entire work week, broken down into ten-minute
increments.   Students are surprised to learn in this assignment that
what might seem like a 12-hour workday is filled with significant
``gaps'' of unproductive activity and interruptions.   

\vspace*{0.1in}
\noindent
{\bf Information management.}  Another difficulty faced by
Ph.D. students is managing the thoughts and ideas that arise when
students read papers, attend talks, meet with their advisors, and so
forth.  We thus include course material managing information, such as
maintaining a research notebook, a research project wiki, and how to
manage email correspondence and notes.  We also introduce tips on how to
have productive advisor meetings, which includes important information
management skills such as coming to the meeting with an agenda, taking
notes on the meeting and sending a post-meeting summary, etc.

\vspace*{0.1in}
\noindent
{\bf Personal development and adjustment.}  Researchers cannot be
productive if they are not happy personally.  Accordingly, we also
include several ``personal development'' aspects in the course.  We
scheduled the class period in the late afternoon to facilitate
transition to community-building social events (e.g., picnics, happy
hours), and incorporated a student panel on living in Atlanta.  We also
incorporated a panel where more senior Ph.D. students shared their
experiences about life in the graduate program.


\subsection{Career Development}

A systematic introduction to employment possibilities early in the
Ph.D. program can help students prepare for their career after graduate
school before it is too late for many options to be viable.  Keeping
certain career options open (e.g., academia) often require building a
solid research reputation over many years.  To help students appreciate
the full range of post-Ph.D. employment possibilities and make
appropriate career choices to help them be marketable when they
graduate, we include course material that explains both job
opportunities and promotion of one's research.

\vspace*{0.1in}
\noindent
{\bf Overviews of job opportunities.}  As previously mentioned, the
second lecture of the course, ``Why a Ph.D.?'' includes a high-level
overview of the job opportunities that either become viable as a result
of having a Ph.D. or are natural consequences of graduate research:
academia, industrial or government research, and entrepreneurship.  We
include specific lectures on life as a professor, the differences
between working in industry vs. academia (including guest speakers from
researchers who have had experience in both areas), and how to
commercialize one's research.  

\vspace*{0.1in}
\noindent
{\bf Personal and research promotion.}  An important factor in a
student's marketability at the time of graduation is how well known the
student and his or her research is in the broader research community.
This oft-overlooked aspect of a student's development as a researcher
can sometimes cost a student opportunities in academic and industry
research positions, where employment opportunities often correlate with
other researchers' familiarity with the student's research work.  To
help students form their research reputations early, the course includes
a lecture on publicizing one's research in the broader community (e.g.,
popular press), as well as lecture material on networking within the
professional community; it also includes mini-assignment where students
must construct a personal research Web page.

\subsection{Exemplars and Bootstrapping}

One of the most difficult aspects of the graduate career is getting
started and, more importantly, appreciating that one's research can, in
fact, have broader impacts.  Thus, a primary goal of the course is to
get students involved in research as soon as possible (i.e.,
immediately) and to appreciate that their work can, in fact, have
broader impacts.  To achieve this goal, we involve students in projects
that incorporate both breadth and depth, and we provide students with
specific examples of computer science research projects that have had
real-world impact.  We also provide basic logistical information to help
students bootstrap at the beginning of their research careers.

\vspace*{0.1in}
\noindent
{\bf Main project and mini-project.}  We require the students to perform
one ``main'' term-long research project (typically with their {\em de
  facto} advisor) and one exploratory mini-project.  The main project
has several benefits: First, it gives students a sense of immediate
accomplishment and defends against students' tendency to feel ``lost''
when they first arrive in the program.  Second, it quickly provides
students with a sense of what it is like to do research; this experience
allows them to quickly discover both the joy and challenge of research,
to appreciate the difference between graduate school and undergraduate
education, to adapt their working skills as needed to be successful, and
to decide whether they like research at all.  The mini-project can be
done with any faculty member in the department and is intended to give
the students an excuse to explore research topics that may be further
afield from their main interest (and, incidentally, might prove useful
in helping them apply cross-disciplinary thinking to their own
research). 

\vspace*{0.1in}
\noindent
{\bf Internal and external speakers.}  The course includes an internal
lecture series to help students familiarize themselves with various
aspects of the department.  Rather than simply providing an overview or
advertisement, however, we asked faculty within the department to
provide a research overview in the form of ``big picture'' research
questions (e.g., What are the five biggest open questions in Computer
Science research? What are current research trends?).  We also include a
lecture from the dean that includes a discussion of broader research
directions, and we have both students and faculty give advice on other
logistical topics, such as how to apply for fellowships.

\vspace*{0.1in}
\noindent
{\bf Broader impact topics.}  Anecdotal evidence suggested that many
Ph.D. students fail to complete their studies because they don't get a
sense that their work has important broad impacts.  To account for this,
we include several lectures in the course on broader impact topics
(e.g., how computing can help people in developing regions, integration
of research with diversity and education).  
\section{Conclusion}\label{sec:conclusion}

While much of the information we present in this course is available in
books and on various websites created by students and various
Ph.D.-holding individuals who have taken interest in certain subsets of
this topic, no single unified and coherent source seems to exist, no
less a structured course.  Our course appears to be the first
programmatic attempt to give students all the unwritten rules for
performing great research and having successful careers in research.
Also, our emphasis on cross-disciplinary work and broader impact brings
these topics stemming from important current trends in science and
society into students' purview at as early a stage as possible.  We
believe these are transformative viewpoints to which students might not
otherwise be exposed at all.  We ultimately hope this course can
serve as a cornerstone for a cross-disciplinary Ph.D. program.

%We are working on a graduate version
%of the Threads initiative which extends the ideas in this course across
%the entire Ph.D. career.
\section{Introduction}

This paper describes a course we developed and taught for the first
time in Fall 2006, CS 7001, as a revamping of the required
introductory course for new students in our Computer Science Ph.D.
program.  We are teaching in Fall 2007 (in progress)
an improved version of the course.

Previously, the course was mainly a semester-long seminar series
intended to present the research of the faculty for
the purposes of advisor selection, with a requirement for short
exploratory ``mini-projects'' for testing the waters with a few faculty
members.  We expanded its scope far beyond advisor selection, to
address a number of observations about current graduate education in
Computer Science:
\begin{itemize}

\item {\bf The general skills of independent research are not taught
programmatically.}  Though diverse topics and research styles exist
within Computer Science, there are universal skills of research and
independent work which could in principle be taught, but are currently
learned in {\it ad hoc}, ill-defined and haphazard fashion, if at all.
In other words, to our knowledge, there is {\it no existing
pedagogical practice for teaching independent research} which is widely
used.
We believe that some of the consequences of the current culture
leaving Ph.D. students to figure out these ``meta'' aspects of
independent research on their own include unnecessary and avoidable
contributions to:
  \begin{itemize}
\itemsep=-1pt
  \item High attrition, or drop-out rates, in Ph.D. programs.
  \item Long graduation times in Ph.D. programs.
  \item Slow ramp-up time toward productivity in research.
  \item Poor job preparation and marketability.
  \item A tendency to follow advisors blindly rather than blaze new directions.
  \end{itemize}

\item {\bf Computer Science research has an opportunity to change all
other disciplines, yet this is not taught.}  Much current CS
research is of an incremental ``stovepipe'' or inward-looking
nature, even though Computer Science has a unique ability to
fundamentally transform virtually every other discipline.  Though
acknowledged often, we do not teach Ph.D. students in CS to actively
seek out and forge the necessary connections to other disciplines.
Unfortunately, to our knowledge, there is {\it no existing
pedagogical practice for teaching cross-disciplinary research} which is
widely used.  We believe that this results in unnecessary contributions to:
  \begin{itemize}
\itemsep=-1pt
  \item Researchers in other disciplines effectively develop their own
        computational solutions, with less expertise.
  \item Untapped potential, in terms of impact and quality of research.
  \end{itemize}

\item {\bf Computer Science has an opportunity for broader appeal,
  inclusion, and societal impact.}  Enrollment in Computer Science
  programs at all levels is affected by perceived societal impact, as
  evidenced by the effect of highly-visible but limited glimpses into
  the impact of CS such as the dot-com economic boom and bust, the
  effect of Google on daily life, and demographic perceptions
  regarding computer scientists.  The reality, that these perceptions
  are limited, needs to be transmitted by the leaders of the field---CS
  Ph.D. students are not taught about the need to change these
  perceptions, inspire, and communicate their impact.  We believe
  this deficiency has resulted in consequences regarding:
  \begin{itemize}
\itemsep=-1pt
  \item Dwindling enrollments in CS Ph.D. programs.
  \item The perception that CS is ``dead'' even though demand is high.
  \item Low enrollments by women and minorities.
  \end{itemize}
\end{itemize}

To correct many of these problems, we have developed a first-of-its-kind
``introduction to graduate research'' course; this course is now in its
second year.  The primary goal of the course is to immerse students in
an environment where they can {\em immediately} become engaged in
high-impact research; a related secondary goal is to teach students the
necessary skills that enable them to quickly ``ramp up'' their research
careers and to set them on a trajectory for fruitful research careers,
as well as marketability.  We have designed a compendium comprising 8
assignments, a research project, a mini-project, and a series of 33
lectures that is organized into modules that give students the
ambition---and the capability---to excel in a graduate research program.
This paper summarizes the high-level goals of the course, how we achieve
these goals with 5 course modules, and our experiences and lessons
teaching the course over the past two years.  Although it might seem
that the course itself contains nothing that is not ``obvious in
hindsight'', feedback and data collected from the first offering of the
course suggest that the course provides significant benefits.



The rest of the paper is organized as follows.
Section~\ref{sec:components} describes the components of the course,
including a high-level overview of the course syllabus.
Section~\ref{sec:progress} describes the progress we have made in
teaching the course to incoming graduate students and discusses various
challenges we encountered, as well as how we addressed them.
Section~\ref{sec:related} describes related work, and
Section~\ref{sec:conclusion} concludes.
% rubber: paper letter

\documentclass{sig-alternate-pretty}
\usepackage{amssymb}
\usepackage{epsfig}
\usepackage{amsmath}
\usepackage{times}
\usepackage{url}
\usepackage{theorem}
\usepackage{subfigure}
\usepackage{algorithm,algorithmic}

\theoremstyle{plain}
 

\widowpenalty 100000
\clubpenalty 100000
  
\newtheorem{defn}{Definition}[section]
\newtheorem{theorem}{Theorem}[section]
\newtheorem{prob}[theorem]{Problem}
\newtheorem{example}[theorem]{Example}
\newtheorem{prop}[theorem]{Proposition}
\newtheorem{lemma}[theorem]{Lemma}
\newtheorem{corollary}[theorem]{Corollary}
\newtheorem{remark}[theorem]{Remark}
\newcommand{\reals}{\ensuremath{\mathbb{R}}}
 
\def\eps{\varepsilon}
\def\R{\reals}
\def\E{{\sf E}}
\def\Pr{{\sf P}}
\def\Var{{\sf Var}}

\newcommand{\ie}{{\em i.e.}}
\newcommand{\eg}{{\em e.g.}}
\newcommand{\ea}{{\em et al. }}
\newcommand{\eans}{{\em et al.}}
 
\long\def\privremark#1{\par\medskip\hrule\medskip{\sl#1}\medskip\hrule\medskip}
%\long\def\privremark#1{}


\begin{document}

\title{Can Great Research Be Taught?  Independent Research
  with Cross-Disciplinary Thinking and Broader Impact}
%A Introductory Ph.D.-Level Course on
%  Cross-Disciplinary Research with Broader Impact} 
%Experiences Teaching Independent
%  Research with Cross-Disciplinary and Broader Impact to Incoming
%  Ph.D. Students\vspace*{-0.25in}}
 
%%Alternate: Reliable Connectivity and Rapid Recovery with Path Splicing
\author{\hspace*{0.5in}Nick Feamster and Alexander Gray \\ \hspace{0.5in}College of Computing, Georgia Tech}
%\author{Authors Removed}
 
\date{}
\maketitle

\begin{sloppypar}

\begin{abstract}
This paper describes a course we have developed for preparing new Ph.D.
students in computer science for a career in research.  The course is
intended to teach the skills needed for research and independent work,
prepare students psychologically and socially for years lying before
them, and help them find a good Ph.D. topic by providing principles and
examples.  In this course, we emphasize and encourage impact through
cross-disciplinary research and broader societal outreach.  To our
knowledge, the course represents a first-of-its-kind systematic
introduction to a graduate research career.  This paper describes our
high-level goals for this curricular initiative, the structure of the
course (including lecture components and assignments), and the
challenges we faced in developing this course.  As we continue to
develop this course, which is now in its second year, we hope it will
serve as a model ``introduction of Ph.D. research'' course for other
computer science departments.
\end{abstract}


\vspace{1mm}
\noindent
{\bf Categories and Subject Descriptors:} K.3.2 {[Computers and
    Education]}: {Computer science education, Curriculum}

\vspace{1mm}
\noindent
{\bf General Terms:} Human Factors, Design

\vspace{1mm}
\noindent
{\bf Keywords:} Ph.D., graduate education, research


\section{Introduction}

This paper describes a course we developed and taught for the first
time in Fall 2006, CS 7001, as a revamping of the required
introductory course for new students in our Computer Science Ph.D.
program.  We are teaching in Fall 2007 (in progress)
an improved version of the course.

Previously, the course was a semester-long seminar series
intended to present the research of the faculty for
the purposes of advisor selection, with a requirement for short
exploratory ``mini-projects'' for testing the waters with a few faculty
members.  We expanded its scope to
address a number of observations about current graduate education in
Computer Science:

{\bf The general skills of independent research are not taught
programmatically.}  Though diverse topics and research styles exist
within Computer Science, there are universal skills of research and
independent work which could in principle be taught, but are currently
learned in {\it ad hoc}, ill-defined and haphazard fashion, if at all.
In other words, to our knowledge, there is {\it no existing
pedagogical practice for teaching independent research} which is widely
used.
We believe that some of the consequences of the current culture
leaving Ph.D. students to figure out these ``meta'' aspects of
independent research on their own include unnecessary and avoidable
contributions to:
  \begin{itemize}
\itemsep=-1pt
  \item High attrition, or drop-out rates, in Ph.D. programs.
  \item Long graduation times in Ph.D. programs.
  \item Slow ramp-up time toward productivity in research.
  \item Poor job preparation and marketability.
  \item A tendency to follow advisors blindly rather than blaze new directions.
  \end{itemize}

{\bf Computer Science research has an opportunity to change all
other disciplines, yet this is not taught.}  Much current CS
research is of an incremental ``stovepipe'' or inward-looking
nature, even though Computer Science has a unique ability to
fundamentally transform virtually every other discipline.  Though
acknowledged often, we do not teach Ph.D. students in CS to actively
seek out and forge the necessary connections to other disciplines.
Unfortunately, to our knowledge, there is {\it no existing
pedagogical practice for teaching cross-disciplinary research} which is
widely used.  We believe that this results in unnecessary contributions to:
  \begin{itemize}
\itemsep=-1pt
  \item Researchers in other disciplines effectively develop their own
        computational solutions, with less expertise.
  \item Untapped potential, in terms of impact and quality of research.
  \end{itemize}

{\bf Computer Science has an opportunity for broader appeal,
  inclusion, and societal impact.}  Enrollment in Computer Science
  programs at all levels is affected by perceived societal impact, as
  evidenced by the effect of highly-visible but limited glimpses into
  the impact of CS such as the dot-com economic boom and bust, the
  effect of Google on daily life, and demographic perceptions
  regarding computer scientists.  The reality, that these perceptions
  are limited, needs to be transmitted by the leaders of the field---CS
  Ph.D. students are not taught about the need to change these
  perceptions, inspire, and communicate their impact.  We believe
  this deficiency has resulted in consequences regarding:
  \begin{itemize}
\itemsep=-1pt
  \item Dwindling enrollments in CS Ph.D. programs.
  \item Perception that CS is ``dead'' even though demand is high.
  \item Low enrollments by women and minorities.
  \end{itemize}

To correct many of these problems, we have developed a first-of-its-kind
``introduction to graduate research'' course; this course is now in its
second year.  The primary goal of the course is to immerse students in
an environment where they can {\em immediately} become engaged in
high-impact research; a related secondary goal is to teach students the
necessary skills that enable them to quickly ``ramp up'' their research
careers and to set them on a trajectory for fruitful research careers,
as well as marketability.  We have designed a compendium comprising 8
assignments, a research project, a mini-project, and a series of 33
lectures that is organized into modules that give students the
ambition---and the capability---to excel in a graduate research program.
This paper summarizes the high-level goals of the course, how we achieve
these goals with 5 course modules, and our experiences and lessons
teaching the course over the past two years.  Although it might seem
that the course itself contains nothing that is not ``obvious in
hindsight'', feedback and data collected from the first offering of the
course suggest that the course provides significant benefits.



The rest of the paper is organized as follows.
Section~\ref{sec:components} describes the components of the course,
including a high-level overview of the course syllabus.
Section~\ref{sec:progress} describes the progress we have made in
teaching the course to incoming graduate students and discusses various
challenges we encountered, as well as how we addressed them.
Section~\ref{sec:related} describes related work, and
Section~\ref{sec:conclusion} concludes.

\section{Course Components}\label{sec:components}

This section describes the five components of our introductory course
to the Ph.D. program:
\begin{itemize}
\itemsep=-1pt
\item {\em Research skills} comprise topics that help students
  recognize, generate, critique and communicate research ideas.
\item {\em Research mechanics} include topics that help students develop
  essential techniques for performing various common tasks in computer
  science research.
\item {\em Skills for independent work} comprise lectures that help
  students develop skills for working independently, including how to stay
  motivated, set goals, and manage time.
\item {\em Career development} is an area students should be be working
  on as early as possible in the Ph.D. program so they can select and
  plan for an appropriate career path.
\item {\em Exemplars and bootstrapping.} The course includes a set of
  lectures to help students become better oriented with computer science
  both within the department and at-large.
\end{itemize}

\subsection{Research Skills}

\begin{table}[t]
\begin{center}
\begin{small}
\begin{tabular}{l|p{2in}}
{\bf Assignment} & {\bf Description} \\ \hline
\multicolumn{2}{c}{{\em Assignments}} \\ \hline
Recognizing good ideas & Read proceedings from top conference in field,
select two ``best'' papers, and provide a research summary and defense
for each. \\
Generating good ideas & Read other students' summaries from first
assignments.  Combine two ideas from first assignment to propose a new
research direction, idea, or solution. \\ 
Critiquing ideas & Read proposals of research ideas from second
assignment, write reviews, and meet in a mock program committee to
select ``best'' proposals. \\
Communicating ideas & Deliver a presentation to the class on a term-long
research project.  \\ \hline
\multicolumn{2}{c}{{\em Mini-Assignments}} \\ \hline
Why Ph.D.? & Write down your goals and expectations for the
Ph.D. program.  Also, summarize the characteristics of the researchers
and research results that you most admire. \\
Time audit & Log how time is spent in 10-minute increments over the
course of a week. \\
Web page & Create a personal research Web page. \\
Elevator pitch & Compose ``elevator pitches'' of various lengths
(30-second, 1-minute,  5-minute) and deliver the 30-second pitch to the
class. 
\end{tabular}
\end{small}
\end{center}
\vspace*{-0.1in}
\caption{Summary of assignments.}
\vspace*{-0.2in}
\label{tab:assignments}
\end{table}

\begin{table}[t]
\begin{center}
\begin{small}
\begin{tabular}{p{2.5in}|r}
{\bf Topic} & {\bf Lectures} \\ \hline
\multicolumn{2}{c}{{\em Research Skills}} \\ \hline
{\bf Background: How research works/Impact} & 1 \\
Recognizing good ideas & 0.5 \\
Generating good ideas & 2 \\
- Problem selection and cross-disciplinary work & \\
- Creativity and idea generation \\
Critiquing ideas & 0.5 \\ 
- Reading and reviewing papers\\
Communicating ideas & 2 \\ 
- How to write a paper \\
- How to give a talk \\ \hline
\multicolumn{2}{c}{{\em Research Mechanics}} \\ \hline
{\bf Background: Research patterns} & 1 \\
Math skills & 1 \\
Data and empirical skills & 1\\
Programming skills & 1 \\
Human-centered research skills & 1 \\ \hline
\multicolumn{2}{c}{{\em Skills for Independent Work}} \\ \hline
{\bf Background: Executing Great Research} & 1 \\
Goal setting & 1 \\
- Why Ph.D.? \\
Motivation & 1 \\
Time management & 1 \\
Information management & 1 \\
Personal development & 2 \\ 
- Student life and social activities & \\
- Graduate school survival skills & \\
\hline
\multicolumn{2}{c}{{\em Career Development}} \\ \hline
Overviews of job opportunities & 3 \\
- Professor life\\
- Industry vs. academia\\
- Commercialization \\
Teaching and TAing & 1 \\
Personal and research promotion (networking, etc.) & 1 \\ \hline
\multicolumn{2}{c}{{\em Exemplars and Bootstrapping}} \\ \hline
Internal speakers (department and area overviews) & 6 \\ 
External speakers & 2 \\
Broader impact & 2 \\
- Computing for social good & \\
- Diversity and women in computing & \\
\end{tabular}
\end{small}
\end{center}
\vspace*{-0.1in}
\caption{Syllabus Overview.}
\vspace*{-0.2in}
\label{tab:syllabus}
\end{table}

The first module, research skills, teaches students how to develop
creativity and critical thinking skills needed to perform research.  We
divide research skills into four main tasks, each of which correspond to
an assignment, as shown in Table~\ref{tab:assignments}.

\vspace*{0.1in}
\noindent
{\bf Recognizing good ideas.}  Helping students recognize good research ideas
is critical for helping them develop taste in research problems, as well
as to help them develop essential skills such as reading papers.  To
help students develop the ability to recognize good research ideas, the
course includes a lecture on reading research papers.  We then ask the
students to put this knowledge into practice in the first assignment,
where they must read the papers from the top conference proceedings in
their field, select two papers from that set of proceedings, summarize
the main ideas from these papers, and defend their choices of these
papers as representing good research problems (e.g., the paper solves a
longstanding open research problem, defines a new field or direction).
This assignment not only helps students develop research taste, but it
also helps them get into the habit of reading research papers early in
their research career.

\vspace*{0.1in}
\noindent
{\bf Generating good ideas.}  Once students are equipped with the the skills
to read conference proceedings and have begun to develop their own
research taste, we help them develop the skills to generate research
ideas.  Of course, there is no formula for creativity, but certain
approaches to research can put students in a position where they are
more likely to have creative thought.  In keeping with our goal to
counter ``stovepipe'' research and foster interdisciplinary thinking, we
encourage students to solve problems that exist at the gap between two
research areas.  To this point, we incorporate several lectures on
``research at the gap'' to help develop cross-disciplinary thinking.

The second assignment encourages students to think about
cross-disciplinary research problems by having them generate a research
idea by combining two ideas selected by students in the first assignment
from different research areas.  The assignment is motivated by the
observation that great progress on problems is often made by applying
ideas or concept from one discipline to a second, seemingly unrelated
problem area.  To save time and survey a broader range of ideas,
students can use the research summaries generated by other students,
but, if they so choose, they can also select their own set of problems.
The output from this assignment is the equivalent of a short research
proposal, comprising the following three aspects: (1)~a concise
description of the problem; (2)~an explanation of why the problem is
important and what practical or broader impacts solving the problem
would have; and (3)~an explanation of why the problem is challenging or
interesting.  This assignment not only helps students gain experience in
creative thinking, but it also gives them valuable experience in
communicating their research ideas and explaining their importance,
skills that are useful for writing papers and, eventually, research
proposals.

\vspace*{0.1in}
\noindent
{\bf Critiquing ideas.}  After learning how to generate research ideas,
the course gives students the opportunity to develop critical thinking
skills by writing reviews of other students' research proposals.  The
course includes lectures on how to critically read and review research
papers and proposals (a skill that is also intimately tied to
recognizing good research ideas and developing taste in research
problems).  The course includes a corresponding assignment to help
students write reviews: each student is given a selection of other
students' research proposals from the second assignment to critically
review.  Based on these reviews, students then form mock program
committees to select the ``best'' research proposals, each of which
receives a cash prize.  This assignment helps students understand
several important aspects of research: First, they develop further
experience critiquing other research ideas.  Second, they gain some
insight into how research ideas are selected by program committees.
This insight into the (sometimes imperfect) review process may provide
some comfort to a student when his or her first paper is rejected.

\vspace*{0.1in}
\noindent
{\bf Communicating ideas.}  Equally important to developing the research
ideas themselves is communicating them clearly to others.  We impress
upon students the importance of communicating research ideas as a
critical step towards having their ideas adopted, applied, or built
upon.  Accordingly, the course has many assignments, such as those
above, that involve developing writing skills.  Additionally, students
have a major writing assignment that is tied to their term-long research
project.  In this assignment, students are asked to write a paper---in
conference-paper format, composition, and style---that summarizes the
term-long research project that they perform throughout the duration of
the term.  In addition to developing writing skills, students also have
a mini-assignment that helps them develop multi-resolution ``elevator
pitch'' talks of various lengths on their research that can be used in
different settings.  These assignments help students develop a
multi-faceted approach to communicating and promoting their research
ideas.



\subsection{Research Mechanics}

Students not only need skills for performing research, they also need to
develop mechanics for performing research tasks that are common across
all disciplines of computer science, particularly for cross-disciplinary
tasks.  We focus on developing mechanics in three areas, devoting a
lecture to each: (1)~math and analytical skills; (2)~programming skills;
and (3)~skills for human-centered research and experimentation.  We
briefly survey the material covered for each of these three topics.

\vspace*{0.1in}
\noindent
{\bf Math skills.} For work involving a mathematical/theoretical
component, many students may only have mathematical knowledge but lack a
number of intuitions which emerge only from research practice.  We will
cover issues such as what constitutes a mathematical "theory", varying
levels of rigor in proofs, various proof approaches and strategies, the
importance of good notation, and idea generation in mathematics.

\vspace*{0.1in}
\noindent
{\bf Data and empirical skills.}  Although some areas of computer
science are more empirical than others, many aspects of research involve
experimental design and data analysis.  Thus, the course includes one
lecture on how to design experiments and draw meaningful conclusions
from experimental data.  We pay close attention to avoiding common
pitfalls with experimental design and data analysis, as well as
clearly presenting experimental results.

\vspace*{0.1in}
\noindent
{\bf Programming skills.}  We devote one lecture to essential
programming skills that are common across computer science research.
This lecture discusses specific research programming skills, such as
version control, rapid prototyping, common design patterns,
documentation for public release of prototypes, and programming skills
for executing reproducible experiments.

\vspace*{0.1in}
\noindent
{\bf Skills for human-centered research.}  Various areas of computer
science ranging from computer networking to human-computer interaction,
require either data that is generated from humans or direct interaction
with human subjects.  To help students prepare for this type of
research, the course includes a lecture on various human-centered
research techniques (e.g., ethnography, discourse analysis), as well as
other logistical issues with human-centered research, such as
institutional review board approval processes.  

%testing human subjects
%... ethnography, discourse analysis

\subsection{Skills for Independent Work}

%Although much research is collaborative in nature, Ph.D. students
%ultimately assume a large amount of responsibility in pushing their own
%research forward.  
Unlike earlier stages of education, Ph.D. research is
largely unstructured and involves setting goals on very long time
horizons.  Accordingly, the course includes lectures and assignments
that help students develop skills to work independently towards
seemingly distant goals.  

\vspace*{0.1in}
\noindent
{\bf Goal setting.}  The course includes lectures on setting goals and
systematically and methodically working towards those goals.  On the
first day of the course, we ask students to answer questions that help
them think about why they have enrolled in a Ph.D. program and what they
hope to achieve by the end of the Ph.D.; we have an accompanying lecture
that discusses the various reasons for obtaining a Ph.D. (e.g., we
discuss the various job options that having a Ph.D. enables).  By
encouraging students to explicitly write down their goals early in the
program, we fend off the potentially dangerous situation where students
enter a Ph.D. program as a default option without a codified set of
goals.  We impress upon students that setting goals early in the program
is important both so that they have something concrete to work towards
and also because long-term goals can help inform priorities and tactical
decisions throughout the program (e.g., how to allocate time to various
research projects, when and how to promote one's research, etc.).

\vspace*{0.1in}
\noindent
{\bf Motivation.} The Ph.D. is a long process with many ups and
downs; sustaining a high level of motivation can be challenging.  As
such, the course includes a lecture on how to stay motivated in the face
of common trials and tribulations of a graduate research career (e.g.,
paper rejections, self-doubt about the direction of one's research,
day-to-day motivation).

\vspace*{0.1in}
\noindent
{\bf Time management.} Incoming Ph.D. students are typically accustomed
to working on well-defined assignments with fixed, near-term deadlines.
In contrast, graduate research typically involves working on loosely
defined problems (at least initially) over significantly longer time
periods.  Without appropriate time management skills, a graduate student
can fall into traps at either of the two extremes: either working all
the time (but not necessarily efficiently) or not working at all until a
deadline is imminent.  Either approach can lead to inefficiency, lack of
productivity, and ultimately lack of sufficient progress, resulting in
attrition or longer graduation times.   To combat these problems, the
course includes a lecture and a mini-assignment on time management.  We
teach students common tips and tricks for managing time and also have
them perform a ``time audit'' assignment, whereby the student records
how he or she spends an entire work week, broken down into ten-minute
increments.   Students are surprised to learn in this assignment that
what might seem like a 12-hour workday is filled with significant
``gaps'' of unproductive activity and interruptions.   

\vspace*{0.1in}
\noindent
{\bf Information management.}  Another difficulty faced by
Ph.D. students is managing the thoughts and ideas that arise when
students read papers, attend talks, meet with their advisors, and so
forth.  We thus include course material managing information, such as
maintaining a research notebook, a research project wiki, and how to
manage email correspondence and notes.  We also introduce tips on how to
have productive advisor meetings, which includes important information
management skills such as coming to the meeting with an agenda, taking
notes on the meeting and sending a post-meeting summary, etc.

\vspace*{0.1in}
\noindent
{\bf Personal development and adjustment.}  Researchers cannot be
productive if they are not happy personally.  Accordingly, we also
include several ``personal development'' aspects in the course.  We
scheduled the class period in the late afternoon to facilitate
transition to community-building social events (e.g., picnics, happy
hours), and incorporated a student panel on living in Atlanta.  We also
incorporated a panel where more senior Ph.D. students shared their
experiences about life in the graduate program.


\subsection{Career Development}

A systematic introduction to employment possibilities early in the
Ph.D. program can help students prepare for their career after graduate
school before it is too late for many options to be viable.  Keeping
certain career options open (e.g., academia) often require building a
solid research reputation over many years.  To help students appreciate
the full range of post-Ph.D. employment possibilities and make
appropriate career choices to help them be marketable when they
graduate, we include course material that explains both job
opportunities and promotion of one's research.

\vspace*{0.1in}
\noindent
{\bf Overviews of job opportunities.}  As previously mentioned, the
second lecture of the course, ``Why a Ph.D.?'' includes a high-level
overview of the job opportunities that either become viable as a result
of having a Ph.D. or are natural consequences of graduate research:
academia, industrial or government research, and entrepreneurship.  We
include specific lectures on life as a professor, the differences
between working in industry vs. academia (including guest speakers from
researchers who have had experience in both areas), and how to
commercialize one's research.  

\vspace*{0.1in}
\noindent
{\bf Personal and research promotion.}  An important factor in a
student's marketability at the time of graduation is how well known the
student and his or her research is in the broader research community.
This oft-overlooked aspect of a student's development as a researcher
can sometimes cost a student opportunities in academic and industry
research positions, where employment opportunities often correlate with
other researchers' familiarity with the student's research work.  To
help students form their research reputations early, the course includes
a lecture on publicizing one's research in the broader community (e.g.,
popular press), as well as lecture material on networking within the
professional community; it also includes mini-assignment where students
must construct a personal research Web page.

\subsection{Exemplars and Bootstrapping}

One of the most difficult aspects of the graduate career is getting
started and, more importantly, appreciating that one's research can, in
fact, have broader impacts.  Thus, a primary goal of the course is to
get students involved in research as soon as possible (i.e.,
immediately) and to appreciate that their work can, in fact, have
broader impacts.  To achieve this goal, we involve students in projects
that incorporate both breadth and depth, and we provide students with
specific examples of computer science research projects that have had
real-world impact.  We also provide basic logistical information to help
students bootstrap at the beginning of their research careers.

\vspace*{0.1in}
\noindent
{\bf Main project and mini-project.}  We require the students to perform
one ``main'' term-long research project (typically with their {\em de
  facto} advisor) and one exploratory mini-project.  The main project
has several benefits: First, it gives students a sense of immediate
accomplishment and defends against students' tendency to feel ``lost''
when they first arrive in the program.  Second, it quickly provides
students with a sense of what it is like to do research; this experience
allows them to quickly discover both the joy and challenge of research,
to appreciate the difference between graduate school and undergraduate
education, to adapt their working skills as needed to be successful, and
to decide whether they like research at all.  The mini-project can be
done with any faculty member in the department and is intended to give
the students an excuse to explore research topics that may be further
afield from their main interest (and, incidentally, might prove useful
in helping them apply cross-disciplinary thinking to their own
research). 

\vspace*{0.1in}
\noindent
{\bf Internal and external speakers.}  The course includes an internal
lecture series to help students familiarize themselves with various
aspects of the department.  Rather than simply providing an overview or
advertisement, however, we asked faculty within the department to
provide a research overview in the form of ``big picture'' research
questions (e.g., What are the five biggest open questions in Computer
Science research? What are current research trends?).  We also include a
lecture from the dean that includes a discussion of broader research
directions, and we have both students and faculty give advice on other
logistical topics, such as how to apply for fellowships.

\vspace*{0.1in}
\noindent
{\bf Broader impact topics.}  Anecdotal evidence suggested that many
Ph.D. students fail to complete their studies because they don't get a
sense that their work has important broad impacts.  To account for this,
we include several lectures in the course on broader impact topics
(e.g., how computing can help people in developing regions, integration
of research with diversity and education).  

\section{Evaluation and Progress}\label{sec:progress}

We collected data on the Fall 2006 offering of the course
from: (1)~a student focus group in February 2007 (which was attended by
students who took the class in Fall 2006 as well as students in previous
offerings); (2)~a town hall meeting with students and faculty in June
2007; (3)~course evaluations; and (4)~many conversations with students
and faculty (in person and over email).

\subsection{Observations}

We observed the following phenomena based on the above feedback and
data:
\begin{itemize}
\itemsep=-1pt
\item Student attendance was often poor.
\item Student contribution to group assignments was very uneven.
\item Senior students were quick to criticize the course when logistical
  and organizational wrinkles occurred.
\item Senior students told new ones to selectively attend lectures.
\item Both students and faculty expressed tension between mini-projects
  (which encourage exploration) and term-long projects (which encourage
  depth). 
\item We had trouble recruiting high-profile speakers to speak in an
  introductory course to first-year Ph.D. students.
\end{itemize}

\subsection{Inferences}

Integrating and distilling the above observations led us to the
following conclusions:
\begin{itemize}
\itemsep=-1pt
%\item There was a tendency not to take course seriously; this was partially
%due to the culture of the old course, and partially because of the 'soft
%skills' nature of the course.
\item Many students lack the context and experience to fully appreciate the
importance of the course material, which largely comprises ``soft''
skills and advice contained in lectures, rather than traditional
testable material.
\item Giving a choice between the mini-projects and a single main project
created confusion and also resulted in a failure to meet our original
goals for redesigning the course (i.e., involving students in in-depth
research projects early in their graduate careers {\em as well as}
encouraging exploration).
\item Due to inertia, our radical changes to the course were met with
  skepticism from older students who had taken the previous version.
%, though once students and
%faculty understood what we were trying to do, there seemed to be broad
%support.  Many older students and faculty in fact told us that they wish
%they had taken a course like ours.
\end{itemize}

\subsection{Corrective Measures}

As we are only now teaching the second offering of the course, is not
yet clear what the impact of the changes will be, but we are confident
that the current offering incorporates many of the suggestions offered
by students and faculty.  Specifically, we are taking the following
corrective measures:
\begin{enumerate}
\itemsep=-1pt
\item We now offer a cash prize for the best research ideas, now take
attendance, changed the required course from pass/fail to letter-graded,
and changed the time of the course to late afternoon to encourage
attendance.
\item We now require students to perform both a main project and a
mini-project, as they both serve critical functions: the former
providing an opportunity to explore a problem in-depth, and the latter
allows students to explore other research areas and meet other faculty
in the department.
\item In the current instantiation, all assignments are done individually,
with the exception of the mock program committee.  Unfortunately, this
leaves only one activity involving teamwork, which is unfortunate given
the ever-increasing importance of teams in research.  We regard this as
an open issue, which we intend to address in future instances of the
course.
\item A number of obvious logistical wrinkles regarding organization, clarity
of the assignments, and scheduling of the course time were ironed out in
the current instance of the course.
\item We are coupling external speaker invitations with department-wide
distinguished lectures.
\end{enumerate}


\section{Related Work}\label{sec:related}

There are a number of works treating or touching upon various elements
of the course, both general and
specific~\cite{phd-not-enough,reshaping-grad-educ,grad-school-survival,
  unwritten-rules,art-sci-investigation,creativity-in-science,res-students-guide}.
For the elements of the course which are not specific to the Ph.D., such
as aspects of time management, there are many popular works.  A number
of works have touched upon the educational side of cross-disciplinary
research~\cite{work-at-boundaries,facilitating-interdisc,
  grand-unif-interdisc,discourse-anal-cross-disc}.  Many
interdisciplinary PhD training initiatives exist for specific topics,
such as mathematical biology, rather than for Computer Science with any
external discipline, or for any discipline in general.  To our knowledge
there exists no other computer science Ph.D. preparation course similar
to ours, which integrates all of the above elements into a single
coherent framework.


\section{Conclusion}\label{sec:conclusion}

While much of the information we present in this course is available in
books and on various websites created by students and various
Ph.D.-holding individuals who have taken interest in certain subsets of
this topic, no single unified and coherent source seems to exist, no
less a structured course.  Our course appears to be the first
programmatic attempt to give students all the unwritten rules for
performing great research and having successful careers in research.
Also, our emphasis on cross-disciplinary work and broader impact brings
these topics stemming from important current trends in science and
society into students' purview at as early a stage as possible.  We
believe these are transformative viewpoints to which students might not
otherwise be exposed at all.  We ultimately hope this course can
serve as a cornerstone for a cross-disciplinary Ph.D. program.

%We are working on a graduate version
%of the Threads initiative which extends the ideas in this course across
%the entire Ph.D. career.
 

\end{sloppypar}

\begin{small}
\bibliographystyle{abbrv}
\bibliography{paper}
\end{small}

\end{document}
\section{Evaluation and Progress}\label{sec:progress}

We collected data on the Fall 2006 offering of the course
from: (1)~a student focus group in February 2007 (which was attended by
students who took the class in Fall 2006 as well as students in previous
offerings); (2)~a town hall meeting with students and faculty in June
2007; (3)~course evaluations; and (4)~many conversations with students
and faculty (in person and over email).

\subsection{Observations}

We observed the following phenomena based on the above feedback and
data:
\begin{itemize}
\itemsep=-1pt
\item Student attendance was often poor.
\item Student contribution to group assignments was very uneven.
\item Senior students were quick to criticize the course when logistical
  and organizational wrinkles occurred.
\item Senior students told new ones to selectively attend lectures.
\item Both students and faculty expressed tension between mini-projects
  (which encourage exploration) and term-long projects (which encourage
  depth). 
\item We had trouble recruiting high-profile speakers to speak in an
  introductory course to first-year Ph.D. students.
\end{itemize}

\subsection{Inferences}

Integrating and distilling the above observations led us to the
following conclusions:
\begin{itemize}
\itemsep=-1pt
%\item There was a tendency not to take course seriously; this was partially
%due to the culture of the old course, and partially because of the 'soft
%skills' nature of the course.
\item Many students lack the context and experience to fully appreciate the
importance of the course material, which largely comprises ``soft''
skills and advice contained in lectures, rather than traditional
testable material.
\item Giving a choice between the mini-projects and a single main project
created confusion and also resulted in a failure to meet our original
goals for redesigning the course (i.e., involving students in in-depth
research projects early in their graduate careers {\em as well as}
encouraging exploration).
\item Due to inertia, our radical changes to the course were met with
  skepticism from older students who had taken the previous version.
%, though once students and
%faculty understood what we were trying to do, there seemed to be broad
%support.  Many older students and faculty in fact told us that they wish
%they had taken a course like ours.
\end{itemize}

\subsection{Corrective Measures}

As we are only now teaching the second offering of the course, is not
yet clear what the impact of the changes will be, but we are confident
that the current offering incorporates many of the suggestions offered
by students and faculty.  Specfically, we are taking the following
corrective measures:
\begin{enumerate}
\itemsep=-1pt
\item We now offer a cash prize for the best research ideas, now take
attendance, changed the required course from pass/fail to letter-graded,
and changed the time of the course to late afternoon to encourage
attendance.
\item We now require students to perform both a main project and a
mini-project, as they both serve critical functions: the former
providing an opportunity to explore a problem in-depth, and the latter
allows students to explore other research areas and meet other faculty
in the department.
\item In the current instantiation, all assignments are done individually,
with the exception of the mock program committee.  Unfortunately, this
leaves only one activity involving teamwork, which is unfortunate given
the ever-increasing importance of teams in research.  We regard this as
an open issue, which we intend to address in future instances of the
course.
\item A number of obvious logistical wrinkles regarding organization, clarity
of the assignments, and scheduling of the course time were ironed out in
the current instance of the course.
\item We are coupling external speaker invitations with department-wide
distinguished lectures.
\end{enumerate}

\section{Related Work}\label{sec:related}

There are a number of works treating or touching upon various elements
of the course, both general and
specific~\cite{phd-not-enough,reshaping-grad-educ,grad-school-survival,
  unwritten-rules,art-sci-investigation,creativity-in-science,res-students-guide}.
For the elements of the course which are not specific to the Ph.D., such
as aspects of time management, there are many popular works.  A number
of works have touched upon the educational side of cross-disciplinary
research~\cite{work-at-boundaries,facilitating-interdisc,
  grand-unif-interdisc,discourse-anal-cross-disc}.  Many
interdisciplinary PhD training initiatives exist for specific topics,
such as mathematical biology, rather than for Computer Science with any
external discipline, or for any discipline in general.  To our knowledge
there exists no other computer science Ph.D. preparation course similar
to ours, which integrates all of the above elements into a single
coherent framework.

